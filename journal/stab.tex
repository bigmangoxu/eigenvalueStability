\documentclass[journal,final]{new-aiaa}
\usepackage[utf8]{inputenc}
\usepackage{color}
\usepackage{algorithm}
\usepackage[noend]{algpseudocode}
\usepackage{graphicx}
\usepackage{amsmath}
\usepackage[version=4]{mhchem}
\usepackage{siunitx}
\usepackage{longtable,tabularx}
\setlength\LTleft{0pt}

\newcommand{\A}{\mathbf{A}}
\newcommand{\B}{\mathbf{B}}
\newcommand{\C}{\mathbf{C}}
\newcommand{\D}{\mathbf{D}}
\newcommand{\E}{\mathbf{E}}
\newcommand{\F}{\mathbf{F}}
\newcommand{\G}{\mathbf{G}}
\newcommand{\HH}{\mathbf{H}}
\newcommand{\I}{\mathbf{I}}
\newcommand{\J}{\mathbf{J}}
\newcommand{\K}{\mathbf{K}}
\newcommand{\LL}{\mathbf{L}}
\newcommand{\PP}{\mathbf{P}}
\newcommand{\R}{\mathbf{R}}
\newcommand{\U}{\mathbf{U}}
\newcommand{\W}{\mathbf{W}}
\newcommand{\w}{\mathbf{w}}

\newcommand{\uu}{\mathbf{u}}
\newcommand{\n}{\mathbf{n}}
\newcommand{\rr}{\mathbf{r}}
\newcommand{\s}{\mathbf{s}}
\newcommand{\e}{\mathbf{e}}
\newcommand{\ddd}{\mathbf{d}}
\newcommand{\f}{\mathbf{f}}
\newcommand{\T}{\mathbf{T}}
\newcommand{\x}{\mathbf{x}}
\newcommand{\y}{\mathbf{y}}
\newcommand{\ttt}{\mathbf{t}}
\newcommand{\bb}{\mathbf{b}}

\newcommand{\vv}{\mathbf{v}}
\newcommand{\boldalpha}{\boldsymbol{\alpha}}


\newcommand{\df}[2]{\dfrac{\partial#1}{\partial#2}}
%\newcommand{\dd}[3]{\dfrac{\partial^2#1}{\partial#2 \partial#3}}
\newcommand{\dd}[3]{{\partial^2#1}/{\partial#2 \partial#3}}
\newcommand{\ff}[2]{\dfrac{d#1}{d#2}}
\newcommand{\ds}[2]{\dfrac{\partial^2#1}{\partial#2^2}}
\newcommand{\dn}[3]{\dfrac{\partial^{#1}#2}{\partial#2^{#1}}}

\newcommand{\dis}{\displaystyle}

\newcommand{\npo}{n\!\!+\!\!1}

\newcommand{\grad}{\textrm{grad\,}}
\newcommand{\Div}{\textrm{div\,}}

\newcommand{\half}{\frac{1}{2}}
\newcommand{\third}{\frac{1}{3}}
\newcommand{\quarter}{\frac{1}{4}}

\newcommand{\Atv}{\A^{\!T}\!v}
\newcommand{\Au}{\A u}
\newcommand{\RK}{R-K\@\xspace}
\newcommand{\vtf}{v^T\!f}
\newcommand{\gtu}{g^T\!u}

\newcommand{\degr}{^{\circ}}
\newcommand{\logt}{\log_{10}}
\newcommand{\dsps}{\displaystyle\strut}

\newcommand{\ind}{\phantom{\bf do}}

\graphicspath{{./pic/}}

\title{Global instability of compressor rotating stall onset}
\author[1]{Shenren Xu	
\footnote{Associate Professor, Yangtze River Delta Research Institute}}
\affil[1]{Northwestern Polytechnical University, Taicang~215400, P.R.~China}
\affil[2]{Beihang University, Beijing 100191, P.R.~China}
\affil[3]{University of Liverpool, Liverpool xxxx, United Kingdom}
\author[2]{Chen He\footnote{Ph.D. Candidate,
School of Energy and Power Engineering; hechen@buaa.edu.cn}}
\author[2]{Dakun Sun\footnote{Associate Professor,
School of Energy and Power Engineering;
sundk@buaa.edu.cn}}
\author[3]{Sebastian Timme\footnote{Senior Lecturer,
School of Engineering; sebastian.timme@liverpool.ac.uk}}	
\author[1]{Dingxi Wang\footnote{Professor,
School of Power and Engery; dingxi\_wang@nwpu.edu.cn}
}

\begin{document}
\maketitle

\begin{abstract}
Rotating stall taking place at the low-mass-flow side of the compressor
map directly determines the compressor's working range

	
Linear global stability analysis is an effective approach to predict the
exact condition at which flow goes unstable. Compared to the time-domain
simulation approach, ??? method can equivalently predict the destabilization
condition, but at a much lower cost, since unsteady simulations are no longer required.
In this work, a Newton--Krylov nonlinear flow solver is used to first solve for the
steady state flow solution and then eigenanalysis is performed by applying the 
implicit-restart Arnoldi method to the exact Jacobian matrix.
By tracking a subset of the eigenspectrum that is close to the imaginary axis,
the least stable eigenmodes can be found. By perturbing the bifurcation parameter,
e.g., the Reynolds number, the Hopf bifurcation point can be identified.
This method is applied to find the critical Reynolds number for a laminar
flow around a circular cylinder above which laminar vortex shedding appears.
Time-accurate unsteady simulation confirms the correctness of the critical eigenvalue
and eigenvector found. It is also applied to a quasi-3D compressor rotor
annular cascade case, for which eigenanalysis is performed and flow physics
is analyzed based on the unstable modes identified. Interesting correlation
between the rotating perturbation pattern and cell rotating speed is found,
which resembles what is observed in experiments.
This work is a first step towards the study of
rotating flow instabilities in turbomachines, such as rotating stall and rotating instability,
and the preliminery results proved promising for future
application to three-dimensional practical problems.
	
	
	
%The compression system in turbomachines, e.g.,  aircraft engines and gas turbines,
%when operating under off-design conditions, exhibits self-excited unsteady phenomena
%such as surge, rotating stall and rotating instability, leading to performance deterioration
%and/or structural damages. Inability to accurately predict when such flow instability occurs
%limits the development of high performance compression system.
%Conventional methods for predicting the onset of such instabilities include analytical,
%semi-analytical-semi-empherical, and unsteady computational fluid dynamics (CFD) simulation,
%which are either fast but less accurate, or more accurate but requires long computational time.
%For engineering application, only steady state CFD analyses are affordable for deployment
%in practicla workflow. 
%In this paper, we explore using the eigenanalysis approach to predict the onset of such
%instabilities. The key benifit of such analysis is that it is capable to capture exactly the
%eigenmodes that destabilize with computational cost of a few steady analyses.
%Pivotal to such analysis is a robust steady state RANS flow solver using the Newton--Krylov
%time-integration scheme which explicitly forms the Jacobian matrix. Once the flow has
%converged fully, the Jacobian matrix is used to compute the relavent eigenvalues and vectors.
%The methodology is applied to the computation of stability boundary of
%(i) the laminar flow around a two-dimensional circular cylinder, and
%(ii) the flow around a quasi-three-dimensional compressor anuular cascade.
%The method developed here has the potential to revive the once-popular
%eigenvalue method for prediction rotating stall and surge, and to provide
%industry with tools to accurately predict the stall line in the early design stage.
\end{abstract}

\section*{Nomenclature}
{\renewcommand\arraystretch{1.0}
\noindent\begin{longtable*}{@{}l @{\quad=\quad} l@{}}
$c_v,c_p$   & specific heat \\
$e$     & internal energy\\
$E$     & total energy\\

$\partial \Omega_r$& boundary of $\Omega_r$
\end{longtable*}}

%{ \color{blue} NASA R67 eigenanalysis to-do list

%\begin{enumerate}
%	\item  {\color{blue}  survey stations 1-2 performance data}
%	\item  {\color{blue} ND=0 eigenvalues for 18kpa through 23kpa}
%	\item  {\color{blue}  single passage URANS for unstable case}
%	\item  {\color{blue}  other ND eigenvalues (phase shift bc)}
%	\item  {\color{blue}  full annulus URANS for unstable case}	
%	
%\end{enumerate}
%}

%{\Large \color{red} Tasks}
%
%\begin{enumerate}
%	\item  {\color{red} Literature review for rotating stall:}
%	\begin{enumerate}
%		\item {\color{red} M-G theory?}
%		\item {\color{red} Subsonic, how many cells? when? roating speed?}	
%		\item {\color{red} Transonic, how many cells? when? rotating speed?}
%		\item {\color{red} Active control achievement so far? low or high speed?}
%		\item{\color{red} Shortcoming of existing theory and applications?}				
%	\end{enumerate}
%	\item  {\color{red} high speed steady performance curve? mesh convergence? validation?
%spectrum? find all ND=1-11? or even 12-22? how to explain 5+17=22?}
%	\item{\color{red} same conclusion for low speed? same conclusions?}
%	\item{\color{red} urans for confirmimg the eigenanalysis?}
%
%\end{enumerate}


\section{Introduction}
%Rotating stall and rotating instability have been studied extensively
%both experimentally~\cite{emmons1955compressor,frank1955propagation,emmons1959survey}
%and numerically~\cite{cornelius2014experimental,he1997computational,vo2008control,pullan2015origins}.
%Early experimental work revealed
%the basic features of such phenemenon and subsequent work has focused
%on building simple analytical models.
%Existing analytical models~\cite{emmons1955compressor,greitzer1986theory} have
%had their success in the early days but the accuracy and effectiveness of them
%is less than sastifactory when applied to realistic configurations and
%more details are needed for quantitative prediction of the stall behavior.
%
%Previous numerical investigations mainly focused on using time-dependent
%unsteady flow analysis using either a fraction or the whole of an annulus.
%A lot of insight into the flow physics for the destablization mechanism
%has been gained from such high fidelity simuations. Unsteady simulations
%are useful for both reproducing the fully destabilized unsteady flows as
%well as for studying the inception of such instability. Due to the high
%computatioal cost of unsteady simulation, it still remains largely as a
%reserach tool to investigate the stall phenomenon on a case-by-case
%basis.
%
%It is widely believed that the fully developed stall and surge behavior is quite
%different from the incipient stall, or pre-stall disturbance~\cite{stenning1980rotating},
%as fully developed rotating stall exhibits strong nonlinearity. However, if the goal is
%to apply active control to suppress the instability at its infancy, then a linear stability
%prediction should suffice, as has been demonstrated in numerous work~\cite{paduano1991active,day1993active,paduano2001compression,day1997stall},
%as the idea is to eliminate the fully-developed rotating cells from forming.
%
%As modern compression systems are designed with higher loading and speed,
%most analytical models proposed in the early days and desmontrated useful
%on low-speed machines are no longer useful as compressibility and complex
%flow mechanism such as boundary-layer-shock-wave interaction becomes
%important. In addition, existing models seldom take into account the exact
%geometry of the blading, and instead, a simple correlation of the compressor
%characteristics is used. This is obviously not desirable as
%geometry details, such as the exact leading edge geometry, have great impact
%on the stall characteristics. This is particularly the case when more complex stall
%phenomenon are considered, such as spike stall, where geometry details
%such as tip gap and the leading edge shape play a major role.
%
%This calls for a stability analysis method based on the three-dimensional
%Reynolds-averaged Navier-Stokes equations, which is regarded as
%the standard industrial tool for predicting steady and unsteady turbomachinery
%performance. In a way, existing stall model needs to be upgraded using the latest
%high-fidelity flow models. Again, either time-dependent simulations or
%steady-state-based eigenanalysis can be used to study the stability based
%on the high-fidelity models and each has its strength. Time-dependent
%analysis is able to capture not only the incipent stall behavior but also the
%details of the transient process, but at a very high computational cost.
%Eigenvalue analysis is a powerful yet inexpensive tool to probe the flow
%near the critical condition, but is still capable of revealing rich flow physics,
%with cost comparable to a few steady state analysis.
%
%In this work, we demonstrate that using eigenanalysis based on a whole-annulus
%steady state solution, the linear stablity demarcation point can be pinpointed
%with the cost of a few steady state analysis, and a full-annulus time-accurate
%unsteady calculation can thus be avoided. This methodology enables quick parametric
%study of the various rotating flow instability phenomenon
%such as rotating stall and rotating instability.
%
%The idea of performing such eigenanalysis is simple. The difficulty is in the detail.
%One common misunderstanding is that an eigenanalysis for large cases is expensive.
%This is true only if we were to compute the full spectrum of a large sparse linear system
%using a direct method~\cite{amestoy2000mumps}.
%However, since a subset of the millions or even billions of eigenvalues are relevant regarding
%the linear stability, typically $\mathcal O(100)$, such eigenanalysis can be done at the cost of
%a few steady state analysis, using iterative eigenvalue calculation
%methods~\cite{sorensen1992implicit,lehoucq1998arpack}.
%
%In practice, such eigenanalysis is rarely done for large, complex cases of industry relevance.
%The challenge is twofold. First,
%in order to perform eigenanalysis, a steady state flow solution should first be
%obtained, requiring the full convergence of the flow solver. While this is easily
%achievable at design condition, obtaining a fully converged solution at off-design
%conditions remains a challange from the perspective of flow solver~\cite{xu2015stabilisation}.
%This is seldom discussed in literature, but widely felt in industry. Secondly, it is
%a common belief that the  computatinal cost of such eigenanlysis is overwhelming
%and is thus impractical for real applications. With the maturing of distributed
%computing, this is no longer the bottleneck and one can easily compute the
%relevant eigenvectors for cases with up to 10 million grid point, and the cost
%only increases linearly with a scalable algorithm. But here the focus shifts slightly
%to the computational methods side from the flow physics . As discussed
%in~\cite{day2016stall}, combining the advancement by computational specialists
%and the expertise from the `stall fraternity' is the right way to advance the research
%in compressor stall study and an effective way to harness better the benifit of
%using CFD. In this paper, we attempt to apply the latest development in large
%scale eigenanalysis computational method to the long-standing problem of
%rotating flow instability, rotating stall in particular, and try to explore the
%underlying flow physics governing the pre- and in-stall behavior, at a
%computational cost that is affordable for industrial applications.
%
%The rest of this paper is organized as follows. First, the basic algorithm of the
%nonlinear flow solver will be discussed in sec.~\ref{lable:sec1}. Fundemantals of 
%performing eigenanalysis based on RANS equations and relevant techniques
%are discussed in sec.~\ref{label:sec2}. Results for the application of eigenanalysis
%to predict flow instability is elaborated in sec.~\ref{label:results} and conclusions
%are drawn in sec.~\ref{conclusion}.



\section{Numerical approach}
\label{solver}
The flow is computed using the NutsCFD solver, whose accuracy has
been validated intensively for axial and centrifugal compressors~\cite{xu2019newton}.
The RANS equations are discretized using the finite volume method on
arbitrary unstructured meshes. The inviscid flux is calculated with Roe's
flux difference splitting scheme and the viscous flux is calculated
using central scheme. Second-order accurate spatial discretisation
is achieved via linear reconstruction of the primitive variable
from the mesh node to the flux face center and the gradient of
flow variables are obtained using the Green-Gauss approach. Simulation
of the unsteady flow for a rotating geometry is converted to the
computation of a steady flow by re-formulating the governing equations
in a reference frame coindending with the rotating geometry.
In order to obtain a steady flow as the base-flow for linearizing
the perturbed flow about and performing eigenvalue analysis, at
physically unstable states, a globalized Newton--Krylov approach
is used to integrate the unsteady RANS equations in time. This is
one of the key technical enablers for performing global instability
analysis for compressors beyond stall.


The nonlinear flow solver used in this work is NutsCFD, an unstructured-mesh
finite-volume RANS solver capable of dealing with rotating frame reference
and periodic boundary conditions.
The solver features the use of the Newton--Krylov algorithm,
which significantly enhances the efficiency and robsutness when
computing turbomachinery flows at off-design conditions.
Details of the solver can be found in
and a brief description of the solution algorithm is provided in this section.

\subsection{Governing equations}
The integral form of the governing equations in a
relative frame of reference with a constant angular
velocity of $\boldsymbol \omega$ is
\begin{equation*}
\dfrac{d}{dt}\int_{\Omega_r} \W dV
+\oint_{\partial \Omega_r} (\F^r_c-\F_v)dS
+\int_{\Omega_r}\F_\omega dV
=0,
\label{governing}
\end{equation*}
where $\W$ are %is
the conservative variables
$\left[\rho,~\rho\uu,~\rho E\right]^T$.
The absolute and relative convective fluxes,
$\F_c$ and $\F^r_c$,
the viscous flux $\F_v$,
and the additional source term due
to rotation, $\F_\omega$,
are defined as %follows
\begin{equation*}
\F_c=
\left [ 
\begin{array}{c}
\rho \uu \cdot \n\\
\rho \uu \uu\cdot \n +  p\n\\
\rho H \uu \cdot \n
\end{array}
\right],
\text{~~}
\F^r_c=
\F_c
-
(\uu_{rot}\cdot \n) 
\left [ 
\begin{array}{c}
\rho\\
\rho \uu\\
\rho E
\end{array}
\right ]
,\text{~~}
\F_v=
\left [ 
\begin{array}{c}
0\\
\tau \cdot \n\\
\uu \cdot \tau \cdot \n + \kappa \n \cdot \nabla T
\end{array}
\right ],\text{~~}
\F_\omega=
\left [ 
\begin{array}{c}
0\\
\rho {\boldsymbol{\omega}} \times \uu\\
0
\end{array}
\right ],
\end{equation*}
with $\uu_{rot}={\boldsymbol \omega} \times \x$.
When $\omega$ is zero, a solver applicable to
non-rotating reference frame is recovered.

Flow is assumed to be fully turbulent and turbulence is
modeled using the negative Spalart--Allmaras
(SA-neg) model~\cite{allmaras2012modifications}.
Compared to the original SA model~\cite{spalart1992one},
this avoids the clipping of the turbulent variable
to a non-negative value which potentially
prevents the full convergence of the nonlinear solver.
The turbulence equation is discretized using
the first-order accurate upwind
scheme~\cite{langer2014agglomeration}.

\subsection{Spatial discretization}
The governing equations are discretized using the
method of lines and thus the spatial and temporal
discretizations can be treated separately.
The governing equations for
the
steady-state solution $\W$
is
\begin{equation}
\label{nonlinear}
\R(\W)={\bf 0},
\end{equation}
where $\R$ is the sum of fluxes and source terms
associated with each control volume. Suppose control
volume %node
$i$ has $N$ flux faces with
area $S_{ik}$ for
$k=1,2,...,N$.
$R_i$ then is 
\begin{equation*}
R_i(\W)=\sum^{N}_{k=1} (\F^r_c-\F_v)S_{ik}+\F_{\omega} V_i,
\end{equation*}
where $V_i$ denotes the volume.
The computation of the convective flux $\F^r_c$ is based on a modification
of the Roe flux scheme to account for the relative reference
frame that is rotating with a constant angular velocity;
while the viscous flux $\F_v$ is the same as in the stationary
reference frame.



\subsection{Temporal discretization for steady solver}
The Newton method solves the steady-state nonlinear
equation~\eqref{nonlinear} iteratively as %follows
\begin{equation*}
\W^{n+1}=\W^n+\beta \Delta \W
\label{newtonTransient}
\end{equation*}
until convergence is reached, i.e., $\|\R(\W)\|=0$,
where $\Delta \W$ is the solution to the linear 
system of equations
\begin{equation*}
\df{\R}{\W}\Delta \W = -\R(\W^n),
\end{equation*}
while $\beta$ is an under-relaxation factor 
obtained using a line search.

Once the spatial discretization,  $\R(\W^n)$, is established,
there are three main steps to complete a Newton update step,
namely, (i) forming the Jacobian matrix, (ii)
solving the large sparse linear system of equations,
and (iii) finding a step size $\beta$ and update the nonlinear
flow solution.
To form the Jacobian matrix, automatic differentiatino tool Tapenade~\cite{Tapenade}
is used, together with graph coloring tool Colpack~\cite{gebremedhin2013colpack}.
By executing the
foward-differentiated residual subroutine for a subsets of nodes
with the same color, the Jacobian matrix is calculated. The resulting
large sparse linear system of equations is solved using GMRES
right-preconditioned by the incomplete LU factorization with zero
fill-in. 

\subsection{Temporal discretization for unsteady solver}




\section{Global linear stability analysis via eigenmode decomposition}
\label{label:sec2}
\subsection{Global linear stability analysis}
%The time-dependent RANS equations, discretized
%using the method of lines,
%is of the following form
%\begin{equation}\label{timedomain}
% \dfrac{d \uu(t)}{dt}=\R(\uu(t))
%\end{equation}
%where $\uu(t)$ is vector of the time-varying conservative flow variables
%and $\R(\uu)$ is the nonlinear residual representating the spatial discretization.
%Note that the residual vector has already the volume of the cell taken into
%account, by diving the sum of the fluxes out of each cell by its volume.
%Assuming a steady state solution $\uu_0$ (equilibrium point of the dynamic system)
%exists, and the time-varying flow variable can be decomposed into the
%steady and the unsteady part 
%\begin{equation*}
%\uu(t):=\uu_0 + \tilde \uu(t)
%\end{equation*}.
%Assuming the perturbation is small, then
%and the governing equation becomes
%\begin{equation*}
%\dfrac{d \tilde \uu(t)}{dt}=A \tilde \uu(t)
%\end{equation*}
%where $A$ is the Jacobian $A:=\df{\R}{\uu}$.
%
%In order to use the eigen model decomposition approach, suppose
%matrix $A$ has right eigenvectors $\{\vvv_i, \vvv_2, ..., \vvv_N\}$,
%which forms the matrix $V$ as its colum vectors. Matrix $A$ can then
%be factorized as 
%\begin{equation*}
%A=V\Lambda V^{-1}
%\end{equation*}
%where the diagonal matrix $\Lambda$ has the eigenvalues of the matrix $A$
%as its diagonal elements.
%Decomposing the unsteady part $\tilde \uu$ in the eigen modal space with
%coordinates $\eta:=\{\eta_1, \eta_2,\dots, \eta_N\}$,
%\begin{equation*}
%\tilde \uu(t) = V \eta(t).
%\end{equation*}
%Substituting $\tilde \uu$ in the governing equation using the eigen modal
%decomposition, it becomes
%\begin{equation*}
%\dfrac{d \eta(t)}{dt}= \Lambda \eta(t).
%\end{equation*}
%All equations are decoupled now and can be written as
%\begin{equation*}
%\dfrac{d \eta_i(t)}{dt}= \lambda_i \eta_i(t)  ~~~\forall i.
%\end{equation*}
%For the linear system to be stable, a sufficient condition is
%that all eigenvalues have negative real parts, i.e., real$(\lambda_i)<0, \forall i$.


%\subsection{Relation with time-accurate unsteady analysis}
%The relation between the eigenvalue and the time-accurate
%unsteady analysis is discussed in this subsection. Assume that
%the backward-differentiation-formula 1 (BDF1) is used.
%The discretized system is then
%\begin{equation}\label{discretetime}
%\dfrac{\eta^{n+1}_i-\eta^n_i}{T} = \lambda_i  \eta_i^{n+1}
%\end{equation}
%which yields
%\begin{equation}\label{discretetime}
%\eta^{n+1}_i   =\dfrac{1}{1-T \lambda_i}  \eta_i^n
%\end{equation}
%which means that for $\eta_i$ to converge to zero,
%the amplification factor has to have a modulus smaller than 1, i.e.,
%\begin{equation*}
%\left |\dfrac{{1}}{1-T \lambda_i} \right  |<1
%\end{equation*}
%or equivalently,
%\begin{equation*}
%\left |1-T \lambda_i \right  |>1
%\end{equation*}
%The stability region is shown in the red part of the plot in 
%
%\begin{figure}[htb]
%	\centering   
%	\includegraphics[width=0.428\textwidth]{PyPlot/eigen-vs-timedomain/bdf1.png}	
%	\caption{123}
%\end{figure}
%
%If BDF2 is used, then
%%%the discretized equation becomes
%\begin{equation}\label{discretetime}
%\dfrac{3\eta^{n+2}_i-4\eta^{n+1}_i +\eta^n}{2T} = \lambda_i  \eta_i^{n+2}
%\end{equation}
%Assuming the amplication factor is $z\in \mathcal C$, and substitute
%$\eta^{n+2}=z^2 \eta^n$ and  $\eta^{n+1}=z \eta^n$ into eqn, one has
%the characteristic polynomial that equals zero:
%\begin{equation*}
%(3+2T\lambda_i) z^2 -4z+1=0
%\end{equation*}
%and the stability region for $T\lambda_i$ can thus be defined as one such
%that $|z|<1$, which is visualized in the figure
%\begin{figure}[htb]
%	\centering   
%	\includegraphics[width=0.4\textwidth]{PyPlot/eigen-vs-timedomain/bdf2.png}	
%	\caption{123}
%\end{figure}

%Based on the stability analysis for the 1-step and 2-step backward differentiation formula (BDF),
%the following conclusions can be drawn regarding its relation to the eigenvalue analysis of the
%Jacobian matrix $A$ for the spatial discretization.
%\begin{enumerate}
%	\item if the matrix $A$ has unstable eigehvalues (those that have positive
%	real parts), then the time-stepping method with a sufficiently small time-step 
%	should also exhibit linear instability (eigenvalues of the timestepping matrix
%	lying outside the unit circle).
%	\item if all eigenvalues of the matrix $A$ have negative real part, then the
%	time-stepping method unconditionally stable, i.e., a small perturbation
%	will decay and diminish.
%\end{enumerate}

%\subsection{Matrix structure for configurations with cyclic symmetry}
%\subsubsection{circulant matrix}
%Consider the simpliest case of a circulant matrix $B$
%\begin{equation*}
%B=circ(b_0,b_1,\dots,b_{M-1})
%=
%\left [
%\begin{array}{c c c c c c }
%b_0 & b_1 & b_2 &  \cdots & b_{M-2} & b_{M-1}\\
%b_{M-1} & b_0 & b_1 &  \cdots & b_{M-3} & b_{M-2}\\
%\vdots & \vdots & \vdots & \vdots & \vdots & \vdots \\
%b_1 & b_2 & b_3 &  \cdots & b_{M-1} & b_{0}
%\end{array}
%\right]
%\end{equation*}
%Due to the cyclic symmetry, it can be verified that matrix $B$ has
%$M$ eigenvalues and eigenvectors
%defined as
%\begin{equation*}
%\vvv_j=
%[1 ,  \rho_j , \rho_j^2,  \dots ,  \rho_{j}^{M-1}]^T,~
%\lambda_j= \sum_{m=0}^{M-1} b_m \rho_j^m~i=0,1,2,\dots, M-1
%\end{equation*}
%Further more, it can be verified that all the eigenvectors are orthogonal to each other.
%\subsubsection{block-circulant matrix}
%A slightly more complex case is when each element in the circulant matrix
%is replaced at a block matrix, e.g., $b_m$ is now an $N\times N$ square matrix,
%resulting in a block-circulant matrix.
%
%Consider a vector $\w$ of the following form:
%%\begin{equation*}
%%\w =
%%\left[
%%\begin{array}{c}
%%\vvv \\
%%\rho \vvv \\
%%\rho^2 \vvv \\
%%\vdots \\
%%\rho^{n-1}\vvv
%%\end{array}
%%\right]
%%\end{equation*}
%\begin{equation*}
%\w =
%\left[
%\vvv ,
%\rho \vvv ,
%\rho^2 \vvv ,
%\dots ,
%\rho^{n-1}\vvv
%\right]^T
%\end{equation*}
%where $\rho$ is any $M$-th root of 1 and $\vvv$ is a vector of length $N$.
%
%In order for $\w$ to be the eigenvector of the block-circulant matrix $B$,
%the following equation has to be satisfied:
%\begin{equation*}
%B \w = \lambda \w
%\end{equation*}
%which can be expanded as follows
%\begin{equation*}
%\begin{array}{rcl}
%(b_0 + \rho b_1+ \rho^2 b_2 + \cdots + \rho ^{M-1}b_{M-1}) \vvv& = & \lambda \vvv\\
%(b_{M-1} + \rho b_0+ \rho^2 b_1 + \cdots + \rho ^{M-1}b_{M-2}) \vvv& = & \lambda\rho  \vvv\\
%(b_{M-2} + \rho b_{M-1}+ \rho^2 b_0 + \cdots + \rho ^{M-1}b_{M-3}) \vvv& = & \lambda\rho^2  \vvv\\
%\dots & = & \dots\\
%(b_1 + \rho b_2+ \rho^2 b_3 + \cdots + \rho ^{M-1}b_{0}) \vvv& = & \lambda\rho^{M-1} \vvv.
%\end{array}
%\end{equation*}
%It can be seen that all rows are identical to the first row, different only by a scalar multiplication.
%$\lambda$ and $\vvv$ can be found by solving the eigenvalue problem of the equation
%in the first row. There are $N$ $\vvv$ for each $\rho$. When $\rho$ varies from $\rho_0$
%to $\rho_{M-1}$, a total of $N\times M$ $\vvv$ can be found. The $(m,n)$-th of them is
%%\begin{equation*}
%%\w_{m,n}=
%%\left[
%%\begin{array}{c}
%%\vvv_{m,n} \\
%%\rho_m \vvv_{m,n} \\
%%\rho_m^2 \vvv_{m,n} \\
%%\vdots \\
%%\rho_m^{M-1}\vvv_{m,n}
%%\end{array}
%%\right]
%%\end{equation*}
%\begin{equation*}
%\w_{m,n}=
%\left[
%\vvv_{m,n} ,
%\rho_m \vvv_{m,n} ,
%\rho_m^2 \vvv_{m,n} ,
%\dots ,
%\rho_m^{M-1}\vvv_{m,n},
%\right]^T
%\end{equation*}
%where $\vvv_{m,n}$ is the $n$-th eigenvalue of 
%\begin{equation*}
%(b_0 + \rho b_1+ \rho^2 b_2 + \cdots + \rho^{M-1}b_{M-1}) \vvv =  \lambda \vvv
%\end{equation*}
%with $\rho=\rho_m$.

%
%\subsubsection{linearized discreted RANS equation in axisymmetric domain}
%The residual vector of the discreted RANS equation $\R$ is linearized with respect
%to the flow variable  $\U$ to arrive at the full Jacobian matrix $\df{\R}{\U}$, which describes the
%dynamic behavior of the system around the equilibrium point. It turns out that the fact that
%the computational domain has a cyclic symmetry can be exploited to reveal some
%characteristics of the eigenspace of the full Jacobian.
%
%To facilate the following discussion, it is assume that the computational mesh, and thus
%the spatial discretization, is cyclic symmetric with $M$-periodicity, with $M$ being the
%number of blades/passages in the entire annulus. Consequently, the full-annulus
%mesh can be divided into $M$ non-overlapping segments, each of which contains
%$N$ grid points. The grid points are ordered in such way that points $1+N\times m$ to  $N+N\times m,~m=0,1,2,\dots, M-1$ belongs to the $m$-th segment. In additional,
%the location of point $n+N\times m1$ (the $n$-th point in segment $m1$) is
%point $n+N\times m2$ (the $n$-th point in segment $m2$) rotated by an angle $(m2-m1)\times 2\pi /M$. The arrangement of the grid point is illustrated in the following figure.
%
%The entire Jacobian matrix then can be partitioned into the following block matrix
%\begin{equation}
%\df{\R}{\U}=
%\left[
%\begin{array}{ccccc}
%\df{\R_0}{\U_0} &\df{\R_0}{\U_1} & \df{\R_0}{\U_2} &\dots & \df{\R_0}{\U_{M-1}}\\   
%\df{\R_1}{\U_0} &\df{\R_1}{\U_1} & \df{\R_1}{\U_2} &\dots & \df{\R_1}{\U_{M-1}}\\   
%\vdots &\vdots& \vdots &\dots & \vdots\\   
%\df{\R_{M-1}}{\U_0} &\df{\R_{M-1}}{\U_1} & \df{\R_{M-1}}{\U_2} &\dots & \df{\R_{M-1}}{\U_{M-1}}
%\end{array}
%\right]
%\end{equation}
%where each block $\df{\R_{m_1}}{\U_{m_2}}, ~m_1,m_2=0,1,\dots, M-1$, is itself an $5N\times 5N$ matrix for Euler/laminar or $6N\times 6N$ matrix for turbulent flow modelled using a one-equation
%turbulence model. It can also be seen that each block is non-zero only when $m_1$ and
%$m_2$ either are identical or differ by 1. However, this property is not exploited in this subsection.
%
%
%By changing the frame of reference to different pitchwise position,
%it can be shown that the matrix can be rewritten as 
%\begin{equation}
%\df{\R}{\U}=
%\left[
%\begin{array}{ccccc}
%\df{\R_0}{\U_0} &\df{\R_0}{\U_1} & \df{\R_0}{\U_2} &\dots & \df{\R_0}{\U_{M-1}}\\   
%B\df{\R_0}{\U_{M-1}} B^{-1}&B\df{\R_0}{\U_0} B^{-1}&B \df{\R_0}{\U_1} B^{-1}&\dots &B \df{\R_0}{\U_{M-2}}B^{-1}\\   
%\vdots &\vdots& \vdots &\dots & \vdots\\   
%B^{M-1}\df{\R_{0}}{\U_1} B^{-(M-1)}&\df{\R_{0}}{\U_2}B^{-(M-1)} & \df{\R_{0}}{\U_3}B^{-(M-1)} &\dots & B^{M-1}\df{\R_{0}}{\U_{0}}B^{-(M-1)}
%\end{array}
%\right]
%\end{equation}
%
%Next, the following change of variable is used:
%\begin{equation*}
%\tilde \U_i = B^{-i} \U_i,~\tilde \R_i = B^{-i} \R_i,~i=0,1,2,\dots,M-1
%\end{equation*}
%The Jacobian then becomes
%%\begin{equation}
%\begin{align}
%\df{\R}{\U}
%&=
%\left[
%\begin{array}{cccc}
%\df{\tilde \R_0}{\tilde \U_0} &
%\df{\tilde \R_0}{\tilde \U_1} B^{-1}&
%\dots &
% \df{\tilde \R_0}{\tilde \U_{M-1}}B^{-(M-1)}\\   
%B\df{\tilde \R_0}{\tilde \U_{M-1}} &
%B\df{\tilde \R_0}{\tilde \U_0} B^{-1}&
%\dots 
%&B \df{\tilde \R_0}{\tilde \U_{M-2}}B^{-(M-1)}\\   
%\vdots &\vdots &\dots & \vdots\\   
%B^{M-1}\df{\tilde \R_{0}}{\tilde \U_1} &
%B^{M-1}\df{\tilde \R_{0}}{\tilde \U_2}B^{-1} &
%\dots &
% B^{M-1}\df{\tilde \R_{0}}{\tilde \U_{0}}B^{-(M-1)}
%\end{array}
%\right]\\
%&=
%\underbrace{
%\left[
%\begin{array}{cccc}
%I & 0  & \cdots & 0\\
%0 & B & \cdots  & 0\\
%\vdots&\vdots &\dots & \vdots\\
%0 &  0 & \cdots & B^{M-1}
%\end{array}
%\right]
%}_{ \mathcal {B}  }
%\underbrace{
%\left[
%\begin{array}{cccc}
%\df{\tilde \R_0}{\tilde \U_0}&
%\df{\tilde \R_0}{\tilde \U_1}&
%\dots &
%\df{\tilde \R_0}{\tilde \U_{M-1}}\\   
%\df{\tilde \R_0}{\tilde \U_{M-1}}&
%\df{\tilde \R_0}{\tilde \U_0}&
%\dots &
%\df{\tilde \R_0}{\tilde \U_{M-2}}\\   
%\vdots&
%\vdots&
%\dots &
%\vdots\\   
%\df{\tilde \R_{0}}{\tilde \U_1}&
%\df{\tilde \R_{0}}{\tilde \U_2}&
%\dots &
%\df{\tilde \R_{0}}{\tilde \U_0}
%\end{array}
%\right]
%}_{ \mathcal {A}  }
%\underbrace{
%\left[
%\begin{array}{cccc}
%I & 0 & \cdots & 0\\
%0 & B^{-1}  & \cdots  & 0\\
%\vdots&\vdots &\dots & \vdots\\
%0 & 0  & \cdots & B^{-(M-1)}
%\end{array}
%\right]
%}_{ \mathcal {B}^{-1}  }
%\\
%&= \mathcal {B}  \mathcal {A} \mathcal {B}^{-1}
%\end{align}
%%\end{equation}
%
%Therefore the Jacobian matrix is similar to a block-circulant matrix $\mathcal{A}$.
%As forementioned, matrix $\mathcal{A}$ has eigenvalues $\lambda$ and eigenvectors
%$\w$. $\w$ is a compound vector formed by assemblying
%$\vvv$ with scalor factor $\rho$. The eigenvector of $\df{\R}{\U}$ can be
%easily obtained by left multiplying the $\w$ with $\mathcal{B}$.
%Finally, the eigenvectors of the Jacobian matrix is
%
%%\begin{equation*}
%%\w_{m,n}=
%%\left[
%%\begin{array}{c}
%%\vvv_{m,n} \\
%%\rho_m B \vvv_{m,n} \\
%%\rho_m^2  B^2\vvv_{m,n} \\
%%\vdots \\
%%\rho_m^{M-1} B^{M-1}\vvv_{m,n}
%%\end{array}
%%\right]
%%\end{equation*}
%
%\begin{equation*}
%\w_{m,n}=
%\left[
%\vvv_{m,n} ,
%\rho_m B \vvv_{m,n} ,
%\rho_m^2  B^2\vvv_{m,n} ,
%\dots ,
%\rho_m^{M-1} B^{M-1}\vvv_{m,n}
%\right]^T
%\end{equation*}
%
%where $\vvv_{m,n}$ is the $n$-th eigenvector of 
%\begin{equation*}
%\left(
%\df{\R_0}{\U_0}+ \rho \df{\R_0}{\U_1}+ \rho^2\df{\R_0}{\U_2}  + \cdots + \rho^{M-1} \df{\R_0}{\U_{M-1}}
%\right )
% \vvv =  \lambda \vvv
%\end{equation*}
%with $\rho=\rho_m$.
%{\color{red} It can be seen that for a given $n$, the $M$ eigenvectors $\w_{0,n},\w_{1,n},\dots,\w_{M-1,n}$
%are modes with nodal diameters from $0$ to $M-1$, or $1-M/2$ to $M/2$,
%provided that $M$ is an even number.}
%

\subsection{Numerical implementation of eigenanalysis}
In theory, performing the global linear stability analysis as described above
is a standard procedure involving three steps: (i) find an equilirium point $\uu_0$;
(ii) linearize the nonlinear residual and form the Jacobian matrix $A$, and
(iii) perform eigenanalysis and find $\Lambda$ and $V$. Step (i) is simply
running the steady state flow solver until a steady state solution
is found. Step (ii) is a by-product of the nonlinear flow calculation using
the NK method, i.e., store away the Jacobian matrix at the final Newton step.
Step (iii) is a bit more involved for high dimensional problems.

For eigenmode computations, the implicitly restarted Arnoldi method proposed
by Sorensen~\cite{sorensen1992implicit} and implemented in the ARPACK
library~\cite{lehoucq1998arpack}, is used
in combination with the NutsCFD solver.
Shift-and-invert spectral transformation is applied to converge to wanted parts
of the eigenspectrum, and critical is therefore the robust solution of many linear
systems of equations. 
Key to efficiently solving the arising large sparse linear system of equations
is the deflated Krylov subspace solver GCRO-DR~\cite{parks2006recycling}.
Compared with the more commonly used GMRES solver~\cite{saad1986gmres},
GCRO-DR is both more CPU time and memory efficient, especially as the system
matrix condition worsens, as demonstrated in~\cite{xu2016enabling,xu2017robust}.

\section{Results}
\label{label:results}

\subsection{Laminar flow vortex shedding}
Steady, unsteady and eigenvalue analysis is performed for two-dimensional
laminar flows around a circular cylinder with Reynolds number between 30
and 100. The computational domain is a $200 m \times 200 m$ square with a
circular cylinder of diameter $D=2m$ centered at the origin.
The computational domain is meshed with quadrilateral elements,
with 51,480 grid points in total. Far-field boundary condition
with total pressure $101325 Pa$, total temperature $288 K$ and
a Mach number of $0.1$ is applied at the left far-field boundary.
Symmetric plane boundary condition is applied at the bottom and
top boundaries. A constant back pressure of $101325 Pa$ is
applied at the right boundary. No-slip viscous wall boundary
condition is applied at the surface of the circular cylinder.
The dynamic viscosity is varied between $\mu=2.779$ and
$\mu=0.8337$Pa$\cdot$s to obtain Reynolds number between
$30$ and $100$.

Steady state flow solutions are obtained using fully implicit
steady solver based on the globalized Newon-Krylov scheme with the
initial Courant number set to $100$, for
Reynolds numbers of $30$, $40$, $45$, $50$, $60$, $80$, and $100$.
Steady state is found when the residual of the continuity equation
is reduced by ten orders of magnitude. The convergence history of
all the steady calculations are shown in Fig.~\ref{fig:cyl-std}.
All solutions are found with 40 nonlinear iterations.


\begin{figure}[htb]
	\centering   
	\includegraphics[width=0.4\textwidth]{cyl-std-conv.pdf}
	\caption{Convergence history of steady flow calculations
		at different Reynolds numbers.}
	\label{fig:cyl-std}
\end{figure}

Eigenanalysis based on the implicitly restarted Arnoldi method
is performed for the each steady flow solution. In particular,
the ARPACK package is used with the shift-invert mode, which can
efficiently calculate a small set of eigenvalues for large system
matrices requiring minimal storage. In order to use the shift-invert
mode, complex shift approximating the main frequency content of the
physical problem is needed, to aim the search for eigenvalues.
For this particular case,, thanks to the prior knowledge that
the Strouhal number ($St$) is around $0.2$, different imaginary
shifts from $0i$ to $1i$ are applied to identify the relevant
eigenvalues. The partial spectra
are shown in Fig.~\ref{fig:cyl-eigen}. The abscissa is labelled
"growth rate", which is the real part of the eigenvalue; while
the ordinate is labelled "Strouhal number", which is converted
from the imaginary part of the eigenvalue, which is also the
angular frequency.

Due to the known symmetry of the spectra for real matrices,
only the upper half of the spectra is computed and shown.
It can be seen that when the parameter $Re$ is varied from
$30$ to $100$, most eigenvalues hardly move, except one
that obviously breaks away from the bulk, and moves fast
towards the positive half of the complex plane. This
particular eigenvalue is marked and highlighted in the
right half of Fig.~\ref{fig:cyl-eigen} to illustrate its
continuous tracing. When this eigenvalue crosses the
imaginary axis (the interpolated $Re$ is approximately
48), the system becomes linearly unstable.

\begin{figure}[htb]
	\centering   
	\includegraphics[width=0.78\textwidth]{cylinder-eigenvalue.png}
	\caption{Eigenvalues for flow around cylinder
		for $Re$ from $30$ to $100$. Onset of instability occurs
	at $Re\approx 48$.  }
	\label{fig:cyl-eigen}
\end{figure}

Focusing on the single mode that destabilizes when $Re$ increases, its
dependence on the parameter $Re$ is shown in Fig.~\ref{fig:cyl-stab},
in terms of both the growth rate and the Strouhal number. The results
are compared with other results from literature, which is also plotted
in the figure. The critical Reynolds number for the bifurcation computed
in our eigenvalue analysis is approximately $Re_{crit}=48$ and the corresponding
Strouhal number is $St_{crit}=0.11673$, which is close to other independently
obtained results. The eigenvalue analysis performed in ~\cite{Crouch2007Predicting} predicts a critical Reynolds number of $Re_{crit}=47$ and a Strouhal number of
$St_{crit}=0.116$. The experimental results obtained in~\cite{C1989Oblique}
is also plotted in Fig.~\ref{fig:cyl-stab}, which although does not
provides the exact critical value, does, however, seems to intercept our
numerical results at a slightly lower Reynolds number of $Re_{crit} \approx 45$.
The deviation of the Strouhal number predicted by the eigenvalue analysis
from the experimental data as the Reynolds number grows beyond the critical
value is not unexpected, as the eigenvalue analysis is linear while the
periodic vortex shedding is nonlinear. Nevertheless, the eigenvalue analysis
accurately predicts the onset of the unsteadiness, both the critical system
parameter, i.e., the Reynolds number, and the angular frequency.

	\begin{figure}[htb]
	\centering   
	\includegraphics[width=0.8\textwidth]{cylinder-stab.png}
	\caption{The growth rate and strouhal number dependence on the
	Reynolds number. Black square: eigenvalue analysis results of
	this work; red square: critical condition from eigenvalue
	analysis results of this work; blue triangle: bifurcation
	point from the eigenvalue analysis in~\cite{Crouch2007Predicting};
	black circle: experimental data from~\cite{C1989Oblique}.}
	\label{fig:cyl-stab}
	\end{figure}

To verify the eigenvalue analysis results, time-accurate
unsteady analysis is performed for flow at $Re=100$.
The unstable steady state for this Reynolds number has a
conjugate pair of destabilizing
eigenvalues $\lambda_{1,2}=1.6245\pm12.39i$, corresponding to a
growth rate of $1.6245$ and an angular frequency of $12.39~rad/s$
or a Strouhal number of $0.116$. This steady state solution is
used to initialize the unsteady simulation. To obtain the unsteady
flow, the Navier--Stokes equations are integrated in time using the second-order backward differentiation formula. The Newton--Krylov
approach is used for the inner iteration, with the Jacobian and preconditioning matrices formed only once at the beginning of the
entire unsteady calculation. A total of 50 subiterations are taken
at each physical time step, ensuring machine error convergence
throughout the entire unsteady run. The physical time step used
is $\Delta T=0.001$ second and the total duration of
the simulation is $25$ seconds. Time-step sensitivity study has been
performed to ensure adequate temporal precision is maintained.
The flow for the unstable steady state (also the solution at $T=0$
for the unsteady run) as well as the flow at the last time step
of the unsteady run are shown in Fig.~\ref{fig:cyl-re100-std-uns}.
The steady flow is reflection symmetric while the unsteady flow
snapshot is non-symmetric and shows the vortex shedding.
\begin{figure}[htb]
	\centering   
	\includegraphics[width=0.8\textwidth]{cyl-flow.png}
	\caption{Left: the velocity magnitude contour plot of the
		converged steady flow for $Re=100$; right:
		the velocity magnitude contour plot of the unsteady
		flow snapshot at $T=25 sec$, exhibiting the
		vortex shedding.
		snapshot at the final step ($T=25$ sec) of
		the unsteady run.}
	\label{fig:cyl-re100-std-uns}
\end{figure}



The lift coefficient during each physical time step is monitored
and plotted for the duration of the unsteady simulation in
Fig.~\ref{fig:cyl-re100-uns}. It can be seen that the lift
coefficient, starting from essentially zero, exponentially
grows for the first over 15 seconds before it levels out and
reach a periodic solution with constant amplitude.
The black solid curve in the left figure is with the form
of $cl_{\text{ROM}}=c_0 e^{\lambda_1 t+i\phi }=
c_0 e^{(1.6245+12.39i) t+i\phi }$ where $c_0$ and $\phi$
are the initial value and a phase shift that need to be
determined via an optimization process that minimizes the
error between the $cl_{\text{ROM}}$ and the linear segment of the
time-domain signal. To examine the frequency information
of the time-domain signal, pseudo-angular frequency is obtained
by taking the peak-to-peak interval of the $cl$ signal is plotted
again time in the right part of Fig.~\ref{fig:cyl-re100-uns}.
Prior to $T=18 sec$, the angular frequency is $12.39 rad/s$
and the Strouhal number is $St=0.116$, corresponding to point
A in Fig.~\ref{fig:cyl-stab}. From $T=18 sec$ to $T=20 sec$, the
angular frequency underwent a transient growth and in just
around 10 periods, it increases to $\omega=17.25 rad/s$ ($St=0.162$),
corresponding to point B in Fig.~\ref{fig:cyl-stab}. This frequency
is very close to the experimental data and thus confirms that it is
due to the non-linearity that the eigenvalue-based frequency diverges
from the unsteady nonlinear flow. The frequency and growth rate of
the unsteady flow in the linear perturbation regime, however, is
accurately predicted by the eigenvalue analysis.


\begin{figure}[htb]
	\centering   
	\includegraphics[width=0.8\textwidth]{frequency-shift.png}	
	\caption{Left: lift coefficient histogram for $Re=100$ computed
	using unsteady solver from a converged unstable steady state,
	compared with the reduced-order-model (ROM) built using the
	unstable eigenmode; right: the pseudo angular frequency computed
	from the peak-to-peak interval of the time-domain lift coefficient
	signal.}
	\label{fig:cyl-re100-uns}
\end{figure}







%\subsection{Singlerow annular cascade (quasi-3D analysis)}
%NutsCFD is used to analyze the performance of the first
%stage rotor (NASA Rotor 67) of a two stage transonic fan
%designed and tested at the NASA Glenn center~\cite{strazisar1989laser}.
%Its design pressure ratio is
%1.63, at a mass flow rate of 33.25 kg/sec. 
%The NASA Rotor 67 has 22 blades with tip radii of 25.7 cm
%and 24.25 cm at the leading and trailing edge, respectively,
%and a constant tip clearance of 1.0 mm. The hub to tip radius
%ratio is 0.375 at the leading edge (TC = 0.6\% span) and 0.478
%at the trailing edge (TC = 0.75\% span). The design rotational
%speed is 16,043 RPM, and the tip leading edge speed is 429 m/s
%with a tip relative Mach number of 1.38.
%
%As a first step, the analysis is performed on the surface of
%revolution taken at approximately 50\% of the blade height.
%The three-dimensional mesh has one cell in the radial direction
%and the subsequent analysis is thus a quasi-3D one.
%
%\subsubsection{Steady state calculation}
%Steady state analysis is first performed for an annular compressor cascade.
%The pressure ratio and efficiency against the mass flow normalized by the
%choked mass flow, produced by incrementally raising the back pressure,
%are shown in Fig.~\ref{fig:r67-performance}.
%The fully-converged steady state flow solution scorresponding to the
%lowest and highest back pressures are shown in Fig.~\ref{fig:r67-flow}.
%
%\begin{figure}[htb]
%	\centering   
%	\includegraphics[width=0.4\textwidth]{r67-2d-50speed-pr.png}
%			~~~~
%	\includegraphics[width=0.4\textwidth]{r67-2d-50speed-eff.png}
%	\caption{}
%	\label{}
%\end{figure}
%
%\begin{figure}[htb]
%	\centering    
%	choked condition \\
%	~\\	
%	\includegraphics[width=0.35\textwidth]{rotor67-2d-rel-mach-0kpa.png}
%	\includegraphics[width=0.35\textwidth]{rotor67-2d-pres-0kpa.png}\\
%%	\includegraphics[width=0.35\textwidth]{rotor67-2d-rel-mach-17d5kpa.png}
%%	\includegraphics[width=0.35\textwidth]{rotor67-2d-pres-17d5kpa.png}	\\
%%	\includegraphics[width=0.35\textwidth]{rotor67-2d-rel-mach-18kpa.png}
%%	\includegraphics[width=0.35\textwidth]{rotor67-2d-pres-18kpa.png}\\
%	~\\
%	near stall condition\\
%	~\\
%	\includegraphics[width=0.35\textwidth]{rotor67-2d-rel-mach-18d5kpa.png}
%	\includegraphics[width=0.35\textwidth]{rotor67-2d-pres-18d5kpa.png}
%	\caption{Pressure (left) and SA variable (right) contours for
%	whole-annulus calculations.}
%	\label{fig:r67-flow}
%\end{figure}
%
%\subsubsection{Eigenanalysis}
%Eigenvalue analysis is performed for all the steady state solutions along the
%speedline. Since only the 'relevant' eigenvalues are desired (relevant here
%means eigenvalues that are likely to across the imaginary axis, corresponding
%to the onset of Hopf-bifurcation), the knowledge that the rotating stall cells
%usually move along the circumferential direction with a speed that is on the
%same order of the rotating rotation speed, each element of the Jacobian
%matrix is therefore divided by a factor of $1/(2\pi \times 16043 / 60)\approx 1/1680$ (assuming
%the rotational speed of 16,043 RPM),
%so that the frequency information represented by the imaginery part of the
%eigenvalues found is in terms of engine order (EO).
%
%For each condition in Fig.~\ref{fig:r67-flow}, the relevant eigenvalues are computed
%by applying shifts of $\sigma= 0+0.5j, 0+1j, 0+1.5j and 0+2j$ to the Jacobian matrices and
%finding the 10 most interior eigenvalues of the shifted matrices.
%
%
%\subsubsection{Interpretation of the eigenanalysis results}
%
%Shown in Fig.~\ref{fig:r67-eigenvalue-18kpa} is a subset of the eigenvalues
%that are near the imaginary axis, which presumally are most likely to
%be unstable. ARPACK is used with various imaginery shifts to compute
%interior eigenvalues. The ones that are suspicious of crossing the
%imaginery axis are shown. A zoomed view of the eigenvalues reveals
%that there are a total of five that have positive real parts, i.e., unstable.
%A single-mode instability is not found for the case most likely because
%the flow condition chosen is one that is deep into the linearly unstable
%region and a bifurcation point should be searched for at a higher flow-rate
%condition. Nevertheless, in the work, we restrict outselves to the analysis
%of this single condition and a thorough exploration of the whole picture will
%be conducted in our future work.
%
%
%\begin{figure}[htb]
%	\centering   
%%	\includegraphics[width=.4\textwidth]{pic/eigenvalue-18kpa.png}
%	\includegraphics[width=\textwidth]{eigen-annular-cascade.png}	
%	\caption{Spectrum for stall condition.}
%	\label{fig:r67-eigenvalue-18kpa}
%\end{figure}
%
%The unstable eigenvector with the smallest imaginary part (lowest point among the
%five unstable eigenvalues) is visualized in Fig.~\ref{fig:r67-eigenvector-18kpa} with
%both the real and imaginary parts. The circumferential shock
%oscillation can be seen. To analyze the spatial modes, data along
%the intersecting curve is taken (marked at the red line in
%Fig.~\ref{fig:r67-eigenvector-18kpa}). This is done for
%each of the 11 modes (marked with red cross in Fig.~\ref{fig:r67-eigenvalue-18kpa}).
%It is clear from the spatial Fouriour analysis
%that each eigenvector corresponds to a rotating pattern with a different
%nodal diameter, which increases from 1 to 11 monotonically from the lowest
%to the highest eigenvalues.
%
%\begin{figure}[htb]
%	\centering   
%	\includegraphics[width=.6\textwidth]{pic/mode-with-nd5.png}
%	\caption{Eigenvector 5 visualized using the real and imaginary parts
%		of energy component}
%	\label{fig:r67-eigenvector-18kpa}
%\end{figure}
%
%%\begin{figure}[htb]
%%	\centering   
%%	\includegraphics[width=.2\textwidth]{pic/mode1-nd1.jpg}~~~~
%%	\includegraphics[width=.2\textwidth]{pic/mode2-nd2.jpg}~~~~
%%	\includegraphics[width=.2\textwidth]{pic/mode3-nd3.jpg}~~~~
%%	\includegraphics[width=.2\textwidth]{pic/mode11-nd11.jpg}			
%%	\caption{The circumferential distribution for the real part of the
%%	pressure component of the 1st, 2nd, 3rd and 11th eigenvectors.}
%%	\label{fig:r67-data-circumferential}
%%\end{figure}
%
%A more involved data processing reveals that the perturbation pattern,
%for every eigenvector, is a travelling wave that rotates in the opposite
%direction of the motion of the rotor (in the relative frame), and with a speed a fraction of the
%shaft rotating speed. This relative rotating speed can be
%calculated using the imaginary of the eigenvalue and the nodal diameter
%of the perturbation pattern as 
%\begin{equation}\label{cellrotspd}
%\Omega_{cellRotation}^{rel}= \dfrac{\text{Imag}(\lambda)}{ND} \Omega_{shaft}
%\end{equation}
%%In experiments, instrumentation for detecting rotating instability is usually
%%sensors installed on the stationary casing and the speed of the rotating
%%cells is also measured in the absolute reference frame.
%In the absolute reference frame, the cell rotating speed (normalized
%with shaft angular frequency) is calculated as
%\begin{equation}\label{cellrotspd}
%\Omega_{cellRotation}^{abs}= \left (1-\dfrac{\text{Imag}(\lambda)}{ND}\right ) \Omega_{shaft}
%\end{equation}
%Applying this formula to each of the 11 eigenmodes leads to the
%correlation between the nodal diameter and the perturbation rotating speed, as shown in Fig.~\ref{fig:r67-eigenvector-18kpa}. Note that we use the terminology 'cell rotating speed'
%to be consistent with the language used by experimentalists when they describe
%the rotating cells. In fact, what is meant in the current context is actually
%`rotating speed of the perturbation pattern'.
%Although it is well-known that the conclusions drawn from a global linear
%stability analysis can not represent the behavior of a saturated limit cycle
%which is highly nonlinear, it seems that the characteristics of the
%rotating perturbation,
%in terms of nodal diameter and rotating speed, is qualitatively representative
%of rotating cells observed experimentally. However, questions such as which eigenmode
%should destabilizes first, and how does the inlet distortion and blade-row interaction in
%multirow configuration affect the conclusion, remain to be answered.
%
%\begin{figure}[htb]
%	\centering   
%{\tiny }	\includegraphics[width=.4\textwidth]{pic/speed-vs-nd.jpg}
%	\caption{The cell rotating speed in the absolute frame of reference v.s. the number of cells (nodal diameter of the perturbation pattern).}
%	\label{fig:r67-eigenvector-18kpa}
%\end{figure}
%
%\subsubsection{Unsteady analysis}
%
%\subsection {NASA Rotor 67}
%{\color{red} Update performance curve with data at
%the correct survey station location.}
%
%\subsubsection{Case overview}
%The final case is NASA Rotor 67, the first-stage rotor of a two-stage transonic fan. It has
%22 low aspect-ratio blades and is designed for a rotational speed of 16,043 rpm,  with a
%total pressure ratio of 1.63 and a mass flow rate of 33.25\,kg/s~\cite{strazisar1989laser}.
%
%%The flow calculations for this case are performed using 72 processors.
%
%\subsubsection{Flow solution validation}
%The computational domain with a single blade passage is meshed
%with 973,065 grid points and 934,400 hexahedral elements. The
%height of the first layer cell off the viscous wall is $10^{-6}$\,m,
%satisfying $y^+\!\approx\!1$. The geometry of the rotor blade as well as
%detailed views of the mesh are shown in Fig.~\ref{rotor67-view}.
%
%\begin{figure}[htb]
%	\centering   
%	\includegraphics[width=0.5\textwidth]{rotor67/rotor67-geo-mesh.png}
%	\caption{NASA rotor 67 blade viewed from the tip (top)
%		and detailed views of the computational mesh (bottom). % for NASA rotor 67.
%		The rotating part of the hub surface, from
%		$x=-1.374$\,cm to $x=9.365$\,cm, is marked with the two vertical lines.}	
%	\label{rotor67-view} 
%\end{figure}
%
%
%At the inlet boundary, a total pressure of $101325~Pa$ and
%a total temperature of $288.15~K$ are imposed.
%The incoming flow is in the axial direction.
%Since the outlet boundary is relatively far from the rotor,
%i.e., the distance from the rotor trailing edge to the outlet
%boundary is over twice the chord length, a constant back
%pressure is imposed at the outlet as the boundary condition.
%Our numerical experiment shows that using a
%radial
%equilibrium boundary condition only changes the resulting
%speedline negligibly.
%%\jcomm{Better to argue with the absence of swirl in the outflow.}
%Therefore, the constant back pressure
%boundary condition is used for simplicity.
%The back pressure is set to 0\,kPa initially and
%then gradually raised to produce the speedline.
%The procedure illustrated in
%Fig.~\ref{radiver-solution-approach1},
%is used to generate the speedline,
%except that for this case, pressure increments of 5\,kPa,
%1\,kPa and 0.1\,kPa are used.
%%The speedline generation process is terminated
%%The numerical stall boundary is found when
%%the steady state solver no longer converges.
%
%The experimental
%and numerical speedlines are shown in Fig.~\ref{rotor67-map}
%for comparison. The relative error for the choked flow
%rate is $1.0\%$, which is similar to reported
%values of other numerical investigations~\cite{Arnone1994Viscous}.
%The under-prediction of the pressure ratio and the efficiency
%is suspected to be caused by the fact that the implementation of
%the S-A turbulence model used here is not fully appropriate
%for this case.
%%inappropriate turbulence model.
%However,
%as this work focuses on the solver convergence and robustness,
%this is not further investigated here. 
%\begin{figure}[htb]
%	\centering   
%	\includegraphics[width=0.3 \textwidth]{rotor67/conv/rotor67-pr.pdf}
%	\hspace{.1\textwidth}
%	\includegraphics[width=0.3 \textwidth]{rotor67/conv/rotor67-eff.pdf}
%	\caption{CFD calculations compared with measurements \cite{Strazisar1989Laser} 
%		for NASA Rotor 67.}
%	\label{rotor67-map} 
%\end{figure}
%
%
%\section{Conclusion}
%\label{conclusion}
%
%Rotating flow instability at near stall condition for an annular compressor cascade is studied using the eigenanalysis approach and the destabilizing eigenmodes are computed and
%analyzed to shed insight on the rotating stall phenomenon. 
%This is the first time a full-order gloabl linear stability analysis based on
%the three dimensional RANS equations is performed to study the destabilising
%mechanism of such turbomachinery flow phenomenon.
%
%Specifically, a stable nonlinear flow solver based on the matrix-forming Newton--Krylov
%approach is used to compute the steady state flow solution at near stall (possibly
%post-stall) condition and the readily available exact Jacobian matrix is then used
%for eigenvalue anlaysis. The eigenanalysis is performed to compute a subset of
%the eigenvalues that are near the imaginary axis, with the implicit-restarted
%Arnoldi method implemented in the ARPACK library. The shift-and-invert
%approach is used to obtain the least unstable eigenvalues.
%
%The methodology is first applied to the classic case of a laminar flow around
%a 2D circular cylidner. By perturbing the system parameter $Re$, Hopf
%bifurcation is identified which is responsible for the inception of the laminar
%vortex shedding. The frequency and linear growth ratio from the eigenanalysis
%agree well with time-dependent simulation during the linear growth regime.
%
%The same procedure is then applied to the a quasi-3d compressor rotor.
%Analysis shows the existance of a complete set of spatial modes that have
%different nodal diameters and
%rotating speeds. These analysis results provide a solid foundation for the
%explanation of various observations in experiments regarding
%rotating flow instabilities. It is revealed that the multiple modes with
%different nodal diameters coexist, as the inherent property of the
%physical system, and it can be hypothesized that the reason for
%different observed stall cell patterns is due to one particular mode being
%excited to finite amplitute first by external disturbance.
%Further more, by processing the spatial modes and the imaginary
%part of the eigenvalues, rotating speeds of the perturbation patterns
%can be calculated and are found to qualitatively agree with the various
%experimentally observed values for rotating stall cells.
%
%The preliminary results presented in this paper represent our first
%attempt to use eigenanalysis based on RANS equations to study
%the rotating flow instability phenomenon in turbomachinary flows.
%The results are promising in that it shows the eigenanalysis method
%is feasible for practical cases and the eigenvectors do capture some
%of the key featuers of the flow instability investigated. However, more in-depth
%study is needed to investigate the bifurcation process for the quasi-3D
%case, and further investigation into three-dimensional cases will be
%carried out in our future work.
%
%%The results also indicate that the incipent rotating stall is dominated by the
%%dynamics of a single complex-conjugate pair of unstable eigenmodes.
%%This provides a clarification of the currently widely spreaded
%%explainations regarding the origin of the rotating stall. In line with the
%%classical explaination of the modal wave instability route to rotating stall,
%%the full-order CFD analysis in this work confirms this, but provides a much
%%better understanding of the detailed flow physics when such instability
%%occurs.
%
%
%\section*{Acknowledgements}
This work received support by the National Natural
Science Foundation of China (Grant No.~51790512).


\bibliography{xu}

\end{document}
