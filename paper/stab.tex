\documentclass[journal,final]{new-aiaa}
\usepackage[utf8]{inputenc}
\usepackage{color}
\usepackage{algorithm}
\usepackage[noend]{algpseudocode}
\usepackage{graphicx}
\usepackage{amsmath}
\usepackage[version=4]{mhchem}
\usepackage{siunitx}
\usepackage{longtable,tabularx}
\setlength\LTleft{0pt}

\newcommand{\A}{\mathbf{A}}
\newcommand{\B}{\mathbf{B}}
\newcommand{\C}{\mathbf{C}}
\newcommand{\D}{\mathbf{D}}
\newcommand{\E}{\mathbf{E}}
\newcommand{\F}{\mathbf{F}}
\newcommand{\G}{\mathbf{G}}
\newcommand{\HH}{\mathbf{H}}
\newcommand{\I}{\mathbf{I}}
\newcommand{\J}{\mathbf{J}}
\newcommand{\K}{\mathbf{K}}
\newcommand{\LL}{\mathbf{L}}
\newcommand{\PP}{\mathbf{P}}
\newcommand{\R}{\mathbf{R}}
\newcommand{\U}{\mathbf{U}}
\newcommand{\W}{\mathbf{W}}
\newcommand{\w}{\mathbf{w}}

\newcommand{\uu}{\mathbf{u}}
\newcommand{\n}{\mathbf{n}}
\newcommand{\rr}{\mathbf{r}}
\newcommand{\s}{\mathbf{s}}
\newcommand{\e}{\mathbf{e}}
\newcommand{\ddd}{\mathbf{d}}
\newcommand{\f}{\mathbf{f}}
\newcommand{\T}{\mathbf{T}}
\newcommand{\x}{\mathbf{x}}
\newcommand{\y}{\mathbf{y}}
\newcommand{\ttt}{\mathbf{t}}
\newcommand{\bb}{\mathbf{b}}

\newcommand{\vv}{\mathbf{v}}
\newcommand{\boldalpha}{\boldsymbol{\alpha}}


\newcommand{\df}[2]{\dfrac{\partial#1}{\partial#2}}
%\newcommand{\dd}[3]{\dfrac{\partial^2#1}{\partial#2 \partial#3}}
\newcommand{\dd}[3]{{\partial^2#1}/{\partial#2 \partial#3}}
\newcommand{\ff}[2]{\dfrac{d#1}{d#2}}
\newcommand{\ds}[2]{\dfrac{\partial^2#1}{\partial#2^2}}
\newcommand{\dn}[3]{\dfrac{\partial^{#1}#2}{\partial#2^{#1}}}

\newcommand{\dis}{\displaystyle}

\newcommand{\npo}{n\!\!+\!\!1}

\newcommand{\grad}{\textrm{grad\,}}
\newcommand{\Div}{\textrm{div\,}}

\newcommand{\half}{\frac{1}{2}}
\newcommand{\third}{\frac{1}{3}}
\newcommand{\quarter}{\frac{1}{4}}

\newcommand{\Atv}{\A^{\!T}\!v}
\newcommand{\Au}{\A u}
\newcommand{\RK}{R-K\@\xspace}
\newcommand{\vtf}{v^T\!f}
\newcommand{\gtu}{g^T\!u}

\newcommand{\degr}{^{\circ}}
\newcommand{\logt}{\log_{10}}
\newcommand{\dsps}{\displaystyle\strut}

\newcommand{\ind}{\phantom{\bf do}}

\graphicspath{{./pic/}}

\title{Rotating flow instability inception prediction via eigen-analysis}
\author[1]{Shenren Xu	
\footnote{ Corresponding author, email address: shenren\_xu@nwpu.edu.cn}}
\affil[1]{Northwestern Polytechnical University, Xi'an 710072, People’s Republic of China}
\affil[2]{Beihang University, Beijing 100191, People’s Republic of China}
\author[2]{Chen He}
\author[1]{Dingxi Wang}

\begin{document}
\maketitle

\begin{abstract}
The compression system in turbomachines, e.g.,  aircraft engines and gas turbines,
when operating under off-design conditions, exhibits self-excited unsteady phenomena
such as surge, rotating stall and rotating instability, leading to performance deterioration
and/or structural damages. Inability to accurately predict when such flow instability occurs
limits the development of high performance compression system.
Conventional methods for predicting the onset of such instabilities include analytical,
semi-analytical-semi-empherical, and unsteady computational fluid dynamics (CFD) simulation,
which are either fast but less accurate, or more accurate but requires long computational time.
For engineering application, only steady state CFD analyses are affordable for deployment
in practicla workflow. 
In this paper, we explore using the eigenanalysis approach to predict the onset of such
instabilities. The key benifit of such analysis is that it is capable to capture exactly the
eigenmodes that destabilize with computational cost of a few steady analyses.
Pivotal to such analysis is a robust steady state RANS flow solver using the Newton--Krylov
time-integration scheme which explicitly forms the Jacobian matrix. Once the flow has
converged fully, the Jacobian matrix is used to compute the relavent eigenvalues and vectors.
The methodology is applied to the computation of stability boundary of
(i) the laminar flow around a two-dimensional circular cylinder, and
(ii) the flow around a quasi-three-dimensional compressor anuular cascade.
The method developed here has the potential to revive the once-popular
eigenvalue method for prediction rotating stall and surge, and to provide
industry with tools to accurately predict the stall line in the early design stage.
\end{abstract}

%\section*{Nomenclature}
%{\renewcommand\arraystretch{1.0}
%\noindent\begin{longtable*}{@{}l @{\quad=\quad} l@{}}
%$c_v,c_p$   & specific heat \\
%$e$     & internal energy\\
%$E$     & total energy\\
%
%$\partial \Omega_r$& boundary of $\Omega_r$
%\end{longtable*}}

\section{{\color{red}Todo list}}

\begin{itemize}
	\item  {\color{red} Investigate the fouriour spectrum
		of the circumferential signal of the eigenvectors and
		interpret the various frequency components. Why ND=5, 17?}
	\item {\color{red} Compare unstable modes with URANS.}
\end{itemize}


\section{Introduction}
Rotating stall and rotating instability has been studied extensively
both experimentally and numerically. Early experimental work revealed
the basic features of such phenemenon and subsequent work has focused
on the modelling of such phenomenon using simple analytical models.
Simple analytical methods have had their success in the early days
but the cost remaims prohibitively high when applied to realistic configurations and
more detail is needed for quantitative prediction of the stall behavior.

However, previous numerical
investigations mainly focused on using time-dependent unsteady
flow analysis using either a fraction or the whole of an annulus.
A lot of insight into the flow physics for the destablization mechanism
has been gained using such high fidelity simuation approaches.
Unsteasy simulation is useful for both reproducing the fully
destabilized unsteady flows as well as for studying the inception
of such instability.

Due to the high computatioal cost of unsteady simulation, it still remains
largely as a reserach tool to investigate the stall phenomenon on a
case-by-case basis and the cost is too high to apply it to daily
design work and use it routinely.

Eigenvalue analysis is a powerful yet inexpensive tool to probe the flow
near the critical condition, revealing rich flow physics with cost comparable
to a steady state analysis. In this work, we demonstrate that using eigenvalue
analysis based on a whole-annulus steady state solution, the linear stablity
demarcation point can be pinpointed with the cost of a few steady state
analysis, and a full-annulus time-accurate unsteady calculation
can thus be avoid. This methodology enables a quick parameter
study to investigate the various rotating flow instability phenomenon
such as rotating stall and rotating instability.

\section{The nonlinear flow solver}
The nonlinear flow solver used in this work is NutsCFD, an unstructured
finite volume Reynolds-averaged Navier--Stokes solver capable of dealing
with rotating frame reference and periodic boundary conditions.
The solver features the use of the Newton--Krylov algorithm,
which significantly enhances the efficiency and robsutness when
computing turbomachinery flows at off-design conditions.
Details of the solver can be found in~{\color{red} ref}
and only general information of the solver theory is provided below.

\subsection{Governing equations}
The integral form of the governing equations in a
relative frame of reference with a constant angular
velocity of $\boldsymbol \omega$ is
\begin{equation*}
\dfrac{d}{dt}\int_{\Omega_r} \W dV
+\oint_{\partial \Omega_r} (\F^r_c-\F_v)dS
+\int_{\Omega_r}\F_\omega dS
=0,
\label{governing}
\end{equation*}
where $\W$ are %is
the conservative variables
$\left[\rho,~\rho\uu,~\rho E\right]^T$.
The absolute and relative convective fluxes,
$\F_c$ and $\F^r_c$,
the viscous flux $\F_v$,
and the additional flux due
to rotation, $\F_\omega$,
are defined as %follows
\begin{equation*}
\F_c=
\left [ 
\begin{array}{c}
\rho \uu \cdot \n\\
\rho \uu \uu\cdot \n +  p\n\\
\rho H \uu \cdot \n
\end{array}
\right],
\text{~~}
\F^r_c=
\F_c
-
(\uu_{rot}\cdot \n) 
\left [ 
\begin{array}{c}
\rho\\
\rho \uu\\
\rho E
\end{array}
\right ]
,\text{~~}
\F_v=
\left [ 
\begin{array}{c}
0\\
\tau \cdot \n\\
\uu \cdot \tau \cdot \n + \kappa \n \cdot \nabla T
\end{array}
\right ],\text{~~}
\F_\omega=
\left [ 
\begin{array}{c}
0\\
\rho {\boldsymbol{\omega}} \times \uu\\
0
\end{array}
\right ],
\end{equation*}
with $\uu_{rot}={\boldsymbol \omega} \times \x$.
When $\omega$ is set to zero, a solver in the
non-rotating reference frame is recovered.4

Turbulence is modeled using the negative Spalart--Allmaras
(SA-neg) model~\cite{allmaras2012modifications}.
Compared to the original SA model~\cite{allmaras2012modifications},
this avoids the clipping of the turbulent variable
to a non-negative value which potentially
prevents the full convergence of the nonlinear solver.
The turbulence equation is discretized using
the first-order accurate upwind
scheme~\cite{langer2014agglomeration}.

\subsection{Spatial discretization}
The governing equations are discretized using the
method of lines and thus the spatial and temporal
discretizations can be treated separately.
The governing equations for
the
steady-state solution $\W$
is
\begin{equation}
\label{nonlinear}
\R(\W)={\bf 0},
\end{equation}
where $\R$ is the sum of fluxes and source terms
associated with each control volume. Suppose control
volume %node
$i$ has $N$ flux faces with
area $S_{ik}$ for
$k=1,2,...,N$.
$R_i$ then is 
\begin{equation*}
R_i(\W)=\sum^{N}_{k=1} (\F^r_c-\F_v)S_{ik}+\F_{\omega} V_i,
\end{equation*}
where $V_i$ denotes the volume.
The computation of the convective flux $\F^r_c$ is based on a modification
of the Roe flux scheme to account for the relative reference
frame that is rotating with a constant angular velocity;
while the viscous flux $\F_v$ is the same as in the stationary
reference frame.



\subsection{Temporal discretization}
The Newton method solves the steady-state nonlinear
equation~\eqref{nonlinear} iteratively as %follows
\begin{equation*}
\W^{n+1}=\W^n+\beta \Delta \W
\label{newtonTransient}
\end{equation*}
until convergence is reached, i.e., $\|\R(\W)\|=0$,
where $\Delta \W$ is the solution to the linear 
system of equations
\begin{equation*}
\df{\R}{\W}\Delta \W = -\R(\W^n) 
\end{equation*}
The under-relaxation factor 
$\beta$ is obtained using
a line search.

Once the spatial discretization,  $\R(\W^n)$, is established,
there are three main steps to complete a Newton update step,
namely, (i) forming the Jacobian matrix, (ii)
solving the large sparse linear system of equations,
and (iii) finding a step size $\beta$ and update the nonlinear
flow solution.
To form the Jacobian matrix, automatic differentiatino tool Tapenade
is used, together with graph coloring tool Colpack. By executing the
foward-differentiated residual subroutine for a subsets of nodes
with the same color, the Jacobian matrix is calculated. The resulting
large sparse linear system of equations is solved using GMRES
right-preconditioned by the incomplete LU factorization with zero
fill-in. 



\section{Global linear stability analysis via eigenmode decomposition}
\subsection{Global linear stability analysis}
A nonlinear dynamic system, e.g., the discretised NS equation discretized
using the method of lines,
has the form
\begin{equation*}
Vol \dfrac{d\uu}{dt}=-\R(\uu)
\end{equation*}
where $\uu$ is the time-varying flow variable and $\R(\uu)$ is the nonlinear
residual representating the spatial discretization.
$Vol$ is a diagonal matrix with the volume of each dual cell on its diagonal.
This term can be eliminated by redefining $\R(\uu)$ by applying volume
scaling to it. The governing equation of the dynamic system then becomes
\begin{equation*}
\dfrac{d\uu}{dt}=-\R(\uu)
\end{equation*}

Assuming a steady state solution $\uu_0$ (equilibrium point of the dynamic system)
exists, and the time-varying flwo variable can be decomposed as the
steady and the unsteady part 
\begin{equation*}
\uu:=\uu_0 + \tilde \uu
\end{equation*}
and the governing equation becomes
\begin{equation*}
\dfrac{d \tilde \uu}{dt}=A \tilde \uu
\end{equation*}
where $A$ is the negative Jacobian $A:=-\df{\R}{\uu}$.

In order to use the eigen model decomposition approach, suppose
the system matrix has right eigenvectors $\{\vv_i, \vv_2, ..., \vv_N\}$,
which forms the matrix $V$ as its colum vectors. Matrix $A$ can then
be factorized as 
\begin{equation*}
A=V\Lambda V^{-1}
\end{equation*}
where the diagonal matrix $\Lambda$ has the eigenvalues on the matrix $A$
as its diagonal elements.
Decomposing the unsteady part $\tilde \uu$ in the eigen modal space,
it can be expressed as a linear combination of all eigenvectors with
coordinates $\eta$
\begin{equation*}
\tilde \uu = V {\mathbf \eta}
\end{equation*}
Substituting the $\tilde \uu$ in the governing equation using the eigen modal
decomposition, it becomes
\begin{equation*}
\dfrac{d \eta}{dt}= \Lambda \eta
\end{equation*}
All variables are decoupled now and it
can be written as
\begin{equation*}
\dfrac{d \eta_i}{dt}= \lambda_i \eta_i  ~~~\forall i
\end{equation*}
For the linear system to be stable, the sufficient and necessary condition is
that all eigenvalues has a negative real part, i.e., real$(\lambda_i)<0, \forall i$.

\subsection{Numerical implementation of eigenanalysis}
In practice, performing the global linear stability analysis as described above
is a standard procedure involving three steps: (i) find an equilirium point $\uu_0$; (ii) linearize the nonlinear residual and form
the Jacobian matrix $A$, and (iii) perform eigenanalysis and find $\Lambda$ and $V$. Step (i) is simply running the steady state flow solver until a steady state solution
is found. Step (ii) is a by-product of the nonlinear flow calculation using
the NK method, i.e., store away the Jacobian matrix at the final Newton step.
Step (iii) is a bit more involved for high dimensional problems.

For eigenmode computations, the implicitly restarted Arnoldi method proposed by Sorensen~\cite{sorensen1992implicit}
and implemented in the ARPACK library~\cite{lehoucq1998arpack}, is used
in combination with the NutsCFD solver.
Shift-and-invert spectral transformation is applied to converge to wanted parts
of the eigenspectrum, and critical is therefore the robust solution of many linear
systems of equations. 
Key to efficiently solving the arising large sparse linear system of equations
is the deflated Krylov subspace solver GCRO-DR~\cite{parks2006recycling}.
Compared with the more commonly used GMRES solver~\cite{saad1986gmres},
GCRO-DR is both more CPU time and memory efficient, especially as the system
matrix condition worsens, as demonstrated in~\cite{xu2016enabling,xu2017robust}.

\section{Results}
\label{results}

\subsection{Laminar flow around a two-dimensional circular cylinder}
Eigenvalue analysis is performed for the canonical case of the
laminar flow around a circular cylinder with the Reynolds number in
the range between 40 and 100. The computational domain is a
circular cylinder centered at the origin with a diameter of $D=10^{-5}$
and the farfield is a circle with a diameter of $100D$. The left half of
the outer circle is set to 'farfield' boundary condition with a incoming
flow of Mach 0.2 in the x-direction, a static pressure of $101325~Pa$ and
and a temperature of $288.15~K$.
The right half ot he circle is set to 'pressure-outlet' boundary condition,
with a constant pressure of $101325~Pa$.
The computational domain is meshed with quadrilateral elements,
with a total of 29600 grid points. The density is $1.225~kg/m^3$.
The dynamic viscosity is varied in order to achieve a particular Reynolds number.

\subsubsection{Steady state calculation}
The steady state flow is obtained by either using an implicit solution method in
Fluent (version 19.2) or by resorting to the Newton-Krylov algorithm in NutsCFD,
despite the fact that the flow is physically unsteady under this condition. The
Mach number contour of the NutsCFD calculation is shown in Fig.~\ref{fig:cyl-re55}.

\begin{figure}[htb]
	\centering   
	\includegraphics[width=0.75\textwidth]{pic/cylinder-std.png}
	\caption{Mach number contour plot of the calculation results by NutsCFD for Re = 55.}
	\label{fig:cyl-re55}
\end{figure}

To compare the Fluent and NutsCFD results quantitatively, the velocity-x
behind the cylinder as well as the pressure coefficient along the cylinder
surface are compared in Fig.~\ref{fig:cyl-re55-u-cp} and very good agreement
can be found.
\begin{figure}[htb]
	\centering   
	\includegraphics[width=0.75\textwidth]{pic/cylinder-std-compare.png}
	\caption{Comparison between Fluent and NutsCFD calculation results for
		velocity-x along the center line behind the cylinder (left) and pressure
		coefficient along the cylinder surface (right).}
		\label{fig:cyl-re55-u-cp}
\end{figure}

\subsubsection{Unsteady calculation}
Experimental results show that the laminar flow around the cylinder
becomes unsteady for Re above a critical value (around 47). To study
this phenomenon, unsteady flows for $Re=55$ and $Re=40$
are performed using {\color{red} NutsCFD}. First, for both conditions, a steady
state flow solution is obtained by converged the residual to machine error.
For $Re=55$, the unsteady simulation is run with the steady state
as initial condition. A BDF2 second-order implicit dual-time-stepping
method is used with the physical time step set to $10^{-8} sec$, that is, $0.01 ms$,
and
the inner loop is solved with a CFL of 1000 and maximum 5 iterations.
Roughly two orders of magnitude of residual drop is achieved for the
inner loop. From Fig.~\ref{fig:cyl-re40-re55-uns}, it can be seen
that after around $100 ms$, the lift coefficient starts to grow and
eventually reaches a saturated limit cycle at around $130 ms$.
On the contrary, running unsteady simulation with a fully converged
steady state for $Re=40$ does not lead to unsteadiness. To probe
the flow at $Re=40$ further, a disturbance is introduced into the
flow from the farfield by setting the incoming flow direction to vertical
for one time step and switching it back to the horizontal direction,
and then continue the unsteady run. The lift coefficient shows a
transient response but eventually slowly delays to zero. These
two sets of lift coefficient signals are plotted in Fig.~\ref{fig:cyl-re40-re55-uns}
in both linear and logrithm scales. The logrithmic plot on the right
clearly shows an exponential growth and decay for $Re=55$ and $Re=40$,
respectively.

\begin{figure}[htb]
	\centering   
	\includegraphics[width=0.49\textwidth]{pic/cl-linear.jpg}	\includegraphics[width=0.49\textwidth]{pic/cl-log.jpg}
	\caption{Lift coeffient histogram for $Re=55$ and $Re=40$.}
	\label{fig:cyl-re40-re55-uns}
\end{figure}

\subsubsection{Eigenanalysis}
An eigenvalue analysis is performed for the steady state solution calculated
in NutsCFD. After converging the steady state solver to machine error ($tol=10^{-14}$),
the exact Jacobian matrix based on the 2nd-order spatial accuracy is calculated and
output to file. The sparse matrix is read into Matlab and eigs is used to compute
a subset of the eigenvalues, with the aim of finding the (hopefully one) unstable mode.
To minimize the computational effort, 10 eigenvalues/vectors are computed for
matrices with different shifts of $0$, $i$, $2i$, $3i$, $4i$, $5i$. All the eigenvalues,
60 in total and with some duplicated, are plotted in Fig.~\ref{fig:cyl-re55-eigen-vs-uns}.
It can be seen that there is one eigenvalue that is on the right side of the imaginary axis,
indicating there is one unstable mode. In the meantime, from the time-domain simulation,
one can extract from the lift-coefficient signal that the flow is linearly growing with a
growth rate of 0.123 and oscillating with a circular frequency of $4.94~rad/s$. This
mode is plotted along with the spectrum and it can be seen that this mode also overlaps
with the unstable eigenvalue from the eigenanalysis. 

\begin{figure}[htb]
	\centering   
	\includegraphics[width=0.9\textwidth]{pic/uns-vs-eigen.png}	
	\caption{Eigenspectrum from the steady state eigenvalue analysis
		compared with the linearly destabiling unsteady simulation for $Re=55$.}
	\label{fig:cyl-re55-eigen-vs-uns}
\end{figure}

The eigenanalysis not only generates the eigenvalues but also the eigenvectors
associated with each eigenvalue. For the unstable mode, the real part of the
density, x/y momentum and energy component of the unstable eigenvector
is shown in Fig.\ref{fig:cyl-re55-eigenmode}.
\begin{figure}[htb]
	\centering   
	\includegraphics[width=0.98\textwidth]{pic/eigenmode-real.png}	
	\caption{Real part of the density, x/y momentum, energy components
		of the unstable eigenmode for $Re=55$.}
	\label{fig:cyl-re55-eigenmode}
\end{figure}

{\color{red} add the eigenvector to the steady state solution to imitate the unsteady snapshot.}

%
%We use the NutsCFD to compute the steady state laminar flow around the
%circular cylinder for Reynolds number 50, 60 and 75. Eigenvalue analysis is
%performed for each case. The three spectra is shown in Fig.~\ref{fig:cyl}.
%It can be seen that linearly unstable mode start to appear from Re=60.
%The unstable mode at Re=60 is visualized on the right in Fig.~\ref{fig:cyl}
%where our results based on the NutsCFD solver is compared to that
%computed using Nektar++~\cite{cantwell2015nektar++}.
%
%\begin{figure}[htb]
%	\centering   
%	\includegraphics[width=0.99\textwidth]{pic/cyl.png}
%	\caption{Left: spectra computed for the laminar flow around circular cylinder
%		at different Reynolds number; right: x- and y-velocity components
%		of the unstable eigenvector computed by NutsCFD compapred with
%		that with Nektar++.}
%	\label{fig:cyl}
%\end{figure}
%
%\subsection{Transonic buffet around a two-dimensional airfoil (NACA0012)}
%
%\subsubsection{Steady state calculation}
%\subsubsection{Unsteady calculation}
%\subsubsection{Eigenvalue analysis}

%\subsection{NACA0012 near transonic buffet onset condition}


\subsection{Transonic flow for an isolated rotor row (quasi-3D analysis)}
NutsCFD is used to analyze the performance of the first
stage rotor (NASA Rotor 67) of a two stage transonic fan
designed and tested at the NASA Glenn center~\cite{strazisar1989laser}.
Its design pressure ratio is
1.63, at a mass flow rate of 33.25 kg/sec. 
The NASA Rotor 67 has 22 blades with tip radii of 25.7 cm
and 24.25 cm at the leading and trailing edge, respectively,
and a constant tip clearance of 1.0 mm. The hub to tip radius
ratio is 0.375 at the leading edge (TC = 0.6\% span) and 0.478
at the trailing edge (TC = 0.75\% span). The design rotational
speed is 16,043 RPM, and the tip leading edge speed is 429 m/s
with a tip relative Mach number of 1.38.

As a first step, the analysis is performed on the surface of
revolution taken at approximately 50\% of the blade height.
The three-dimensional mesh has one cell in the radial direction.

\subsubsection{Steady state calculation}
Steady state analysis is performed for both
the single-passage and whole-annulus
configurations. In order to obtain the steady state
solutions for the whole annulus, which presumably
is identical for each blade passage, we first compute
the steady state solution for one passage with
rotatioal periodicity, and then copy the solution
to the whoe annulus using rotational transformation.
The pressure ratio and efficiency are shown in Fig.~\ref{fig:r67-performance}
which is produced by incrementally raising the back pressure
from the inlet total condition. It can be seen that there is small
difference (mainly efficiency) between the single-passae and
whole-annulus results, which is due to the minor discrepancy
of the spatial discretization at the periodic boundaries for single
passage calculation.

\begin{figure}[htb]
	\centering   
	\includegraphics[width=0.9\textwidth]{pic/rotor67-performance.png}
	\caption{Q3D performance for rotor67 at 50\% blade height with either
		single passage or whole annulus (22 passages).}
	\label{fig:r67-performance}
\end{figure}

The flow solution using either single passage or whole annulus
is shown in Fig.~\ref{fig:r67-flow}, which visually shows that
the solutions are not distinguishable.

\begin{figure}[htb]
	\centering   
	\includegraphics[width=0.45\textwidth]{pic/pressure-1passage.jpeg}
	\includegraphics[width=0.45\textwidth]{pic/sa-1passage.jpeg}\\    \includegraphics[width=0.45\textwidth]{pic/pressure-22passage.jpeg}
	\includegraphics[width=0.45\textwidth]{pic/sa-22passage.jpeg}
	\caption{Pressure (left) and SA variable (right) contours for
	single passage (top) and whole annulus (bottom) calculations.}
	\label{fig:r67-flow}
\end{figure}

\subsubsection{Eigenanalysis}

For each whole-annulus steady state solution, eigenanalysis
is performed using the Jacobian matrix output from the NutsCFD solver
once the steady state calculation has fully converged.
Since the rotational speed for the rotor is 16,043 RPM, the Jacobian
matrix is scaled by a factor of $1/(2\pi \times 16,043 / 60)\approx 1/1680$,
so that all frequencies involved in this computation is reduced by the
rotor angular frequency. This is done due to the pre-knowledge that
rotating stall cells move with speed of the same order of magnitude.

Shown in Fig.~\ref{fig:r67-eigen-18kpa} is a subset of the eigenvalues
that are near the imaginary axis, which presumally are most likely to
be unstable. The eigs function in Matlab is used with various
imaginery shifts to compute interior eigenvalues. The ones that
are suspecious of crossing the imaginery axis are shown. A zoomed
view of the eigenvalues reveals that there are a total of five that
have positive real parts, i.e., unstable.


The pressure component of the eigenvector corresponding to one
eigenvalue is visualized to the right of Fig.~\ref{fig:r67-eigenvalue-18kpa}. The
circumferential synchronized shock oscillation can be easily spotted.
Eigenvectors of other unstable mode are similar, except with different
circumferential variation, which can further be attributed to different
nodal diameters, which varies from 5 to 9, continuously, from eigenvalues
1 to 5.


\begin{figure}[htb]
	\centering   
	\includegraphics[width=.9\textwidth]{pic/rotor67-2d-eigenvalue-18kpa.png}
	\caption{Spectrum for back pressure 18kpa.}
	\label{fig:r67-eigenvalue-18kpa}
\end{figure}

The eigenvector labelled with '1' is visualized in
Fig.~\ref{fig:r67-eigenvector-18kpa} with both the
real and imaginary parts. The synchronized circumferential shock
oscillation can be seen. To analyze the spatial modes, data along
the intersecting curve is taken (marked at the red line in
Fig.~\ref{fig:r67-eigenvector-18kpa}). This is done for
each of the 11 modes, and a few of them are shown in
Fig.~\ref{data-circumferential}. It is clear that the $n$-th eigenvector
corresponds to a rotating pattern with nodal diameter $n$.

\begin{figure}[htb]
	\centering   
	\includegraphics[width=.9\textwidth]{pic/mode-with-nd5.png}
	\caption{Eigenvector 5 visualized using the real and imaginary parts
		of the pressure.}
	\label{fig:r67-eigenvector-18kpa}
\end{figure}

\begin{figure}[htb]
	\centering   
	\includegraphics[width=.2\textwidth]{pic/mode1-nd1.jpg}~~~~
	\includegraphics[width=.2\textwidth]{pic/mode2-nd2.jpg}~~~~
	\includegraphics[width=.2\textwidth]{pic/mode3-nd3.jpg}~~~~
	\includegraphics[width=.2\textwidth]{pic/mode11-nd11.jpg}			
	\caption{The circumferential distribution for the real part of the
	pressure component of the 1st, 2nd, 3rd and 11th eigenvectors.}
	\label{fig:r67-data-circumferential}
\end{figure}

A more involved data processing reveals that the perturbation pattern,
for every eigenvector, is a travelling wave that rotates in the opposite
direction of the motion of the rotor, and with a speed a fraction of the
shaft rotating speed. This relative rotating speed can be
calculated using the imaginary of the eigenvalue and the nodal diameter
of the perturbation pattern as $\dfrac{\omega}{\omega_{shaft}} \dfrac{1}{ND}$.
In experiment, instrumentation for detecting rotating instability is usually
sensors installed on the stationary casing and the speed of the rotating
cells is also measured in the absolute reference frame. In the
absolute reference frame, the cell rotating speed is calculated as
\begin{equation}\label{cellrotspd}
U_{cellRotating}= 1-\dfrac{\omega}{\omega_{shaft}} \dfrac{1}{ND}.
\end{equation}
Applying this formula to each of the 11 eigenmodes leads to the
correlation between the nodal diameter and the perturbation rotating speed.

\begin{figure}[htb]
	\centering   
{\tiny }	\includegraphics[width=.6\textwidth]{pic/speed-vs-nd.jpg}
	\caption{The cell rotating speed v.s. the number of cells (nodal diameter of the perturbation pattern).}
	\label{fig:r67-eigenvector-18kpa}
\end{figure}

%\begin{table}[]
%	\centering 
%		\caption{Rotating pattern travelling speed.}
%	\begin{tabular}{|c|c|c|c|}
%		\hline
%		eigenmode \# & $\omega$/$\omega_{shaft}$ & Nodal diameter & Cell rotating speed (absolute frame) \\ \hline
%		1            & 0.5774     & 1       & 0.4226      \\ \hline
%	    2            & 0.6604     & 1       & 0.6698      \\ \hline				
%		3            & 0.7885     & 3       & 0.7372      \\ \hline
%		4           & 0.87093     & 4       & 0.7824      \\ \hline				
%		5            & 0.9341     & 5       & 0.813      \\ \hline
%		6            & 0.992      & 6       & 0.835     \\ \hline
%		7            & 1.0427     & 7       & 0.851     \\ \hline
%		8            & 1.0909    & 8       & 0.8636     \\ \hline
%		9            & 1.1329        & 9       & 0.8741     \\ \hline
%	  10            & 1.1736        & 9       & 0.8826     \\ \hline
%	  11            & 1.2165        & 9       & 0.8894    \\ \hline	
%\end{tabular}
%\label{table1}
%\end{table}

\section{Conclusion}
\label{conclusion}

\section*{Acknowledgements}

\bibliography{xu}

\end{document}
