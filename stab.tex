\documentclass[journal,final]{new-aiaa}
\usepackage[utf8]{inputenc}
\usepackage{color}
\usepackage{algorithm}
\usepackage[noend]{algpseudocode}
\usepackage{graphicx}
\usepackage{amsmath}
\usepackage[version=4]{mhchem}
\usepackage{siunitx}
\usepackage{longtable,tabularx}
\setlength\LTleft{0pt}

\newcommand{\A}{\mathbf{A}}
\newcommand{\B}{\mathbf{B}}
\newcommand{\C}{\mathbf{C}}
\newcommand{\D}{\mathbf{D}}
\newcommand{\E}{\mathbf{E}}
\newcommand{\F}{\mathbf{F}}
\newcommand{\G}{\mathbf{G}}
\newcommand{\HH}{\mathbf{H}}
\newcommand{\I}{\mathbf{I}}
\newcommand{\J}{\mathbf{J}}
\newcommand{\K}{\mathbf{K}}
\newcommand{\LL}{\mathbf{L}}
\newcommand{\PP}{\mathbf{P}}
\newcommand{\R}{\mathbf{R}}
\newcommand{\U}{\mathbf{U}}
\newcommand{\W}{\mathbf{W}}
\newcommand{\w}{\mathbf{w}}

\newcommand{\uu}{\mathbf{u}}
\newcommand{\n}{\mathbf{n}}
\newcommand{\rr}{\mathbf{r}}
\newcommand{\s}{\mathbf{s}}
\newcommand{\e}{\mathbf{e}}
\newcommand{\ddd}{\mathbf{d}}
\newcommand{\f}{\mathbf{f}}
\newcommand{\T}{\mathbf{T}}
\newcommand{\x}{\mathbf{x}}
\newcommand{\y}{\mathbf{y}}
\newcommand{\ttt}{\mathbf{t}}
\newcommand{\bb}{\mathbf{b}}

\newcommand{\vv}{\mathbf{v}}
\newcommand{\boldalpha}{\boldsymbol{\alpha}}


\newcommand{\df}[2]{\dfrac{\partial#1}{\partial#2}}
%\newcommand{\dd}[3]{\dfrac{\partial^2#1}{\partial#2 \partial#3}}
\newcommand{\dd}[3]{{\partial^2#1}/{\partial#2 \partial#3}}
\newcommand{\ff}[2]{\dfrac{d#1}{d#2}}
\newcommand{\ds}[2]{\dfrac{\partial^2#1}{\partial#2^2}}
\newcommand{\dn}[3]{\dfrac{\partial^{#1}#2}{\partial#2^{#1}}}

\newcommand{\dis}{\displaystyle}

\newcommand{\npo}{n\!\!+\!\!1}

\newcommand{\grad}{\textrm{grad\,}}
\newcommand{\Div}{\textrm{div\,}}

\newcommand{\half}{\frac{1}{2}}
\newcommand{\third}{\frac{1}{3}}
\newcommand{\quarter}{\frac{1}{4}}

\newcommand{\Atv}{\A^{\!T}\!v}
\newcommand{\Au}{\A u}
\newcommand{\RK}{R-K\@\xspace}
\newcommand{\vtf}{v^T\!f}
\newcommand{\gtu}{g^T\!u}

\newcommand{\degr}{^{\circ}}
\newcommand{\logt}{\log_{10}}
\newcommand{\dsps}{\displaystyle\strut}

\newcommand{\ind}{\phantom{\bf do}}

\graphicspath{{./pic/}}

\title{Rotating flow instability prediction using eigenvalue analysis}
\author[1]{Shenren Xu	
\footnote{ Email address: shenren\_xu@nwpu.edu.cn}}
\affil[1]{School of Power and Energy, 
	Northwestern Polytechnical University, Xi'an, 710072, China}
\author[1]{Zhihao Wu}
\author[3]{Xiuquan Huang}
\author[4]{Dingxi Wang}

\begin{document}
\maketitle

\begin{abstract}
%The compression system in turbomachines, e.g.,  aircraft engines and gas turbines,
%when operating under off-design conditions, exhibits flow instability such as surge,
%rotating stall leading to performance deterioration, noises and structural damages.
%Inability to accurately predict the condition under which such flow instability will
%happen severe hinders the development of high performance compression system.
%In this work,  an eigenvalue analysis method is developed to predict flow stability
%commonly seen for turbomachines operating at near stall conditions. The eigenvalue
%analysis is fully based on the steady state three-dimensional Reynods-averaged
%Navier-Stokes equations and thus the stability boundary is fully consistent with
%the one that is predicted by the time-accurate flow simulation, i.e., URANS, but
%two to three times faster.
%The method is applied to the computation of stability boundary of
%(i) the laminar flow around a two-dimensional circular cylinder,
%(ii) the flow around a quasi-three-dimensional compressor anuular cascade, 
%and 
%(iii) flow around a three-dimensional compressor rotor.
%The method developed here has the potential to revive
%the once-popular eigenvalue method for prediction rotating stall and surge, which
%was based on lower-fidelity flow models and provide industry with tools to
%accurately predict the stall line in the early design stage.
\end{abstract}

%\section*{Nomenclature}
%{\renewcommand\arraystretch{1.0}
%\noindent\begin{longtable*}{@{}l @{\quad=\quad} l@{}}
%$c_v,c_p$   & specific heat \\
%$e$     & internal energy\\
%$E$     & total energy\\
%
%$\partial \Omega_r$& boundary of $\Omega_r$
%\end{longtable*}}

\section{The nonlinear flow solver}
%The nonlinear flow solver used in this work is NutsCFD, an unstructured
%finite volume Reynolds-averaged Navier--Stokes solver capable of dealing
%with rotating frame reference and periodic boundary conditions.
%The solver features the use of a noval Newton--Krylov algorithm,
%which significantly enhances the efficiency and robsutness when
%computing turbomachinery flows at off-design conditions.
%Details of the NK algorithm can be found in~\cite{123} and
%only general information of the solver theory is provided below.

\subsection{Governing equations}
%The integral form of the governing equations in a
%relative frame of reference with an angular
%velocity of $\boldsymbol \omega$ is
%\begin{equation*}
%\dfrac{d}{dt}\int_{\Omega_r} \W dV
%+\oint_{\partial \Omega_r} (\F^r_c-\F_v)dS
%+\int_{\Omega_r}\F_\omega dS
%=0,
%\label{governing}
%\end{equation*}
%where $\W$ are %is
%the conservative variables
%$\left[\rho,~\rho\uu,~\rho E\right]^T$.
%The absolute and relative convective fluxes,
%$\F_c$ and $\F^r_c$,
%the viscous flux $\F_v$,
%and the additional flux due
%to rotation, $\F_\omega$,
%are defined as %follows
%\begin{equation*}
%\F_c=
%\left [ 
%\begin{array}{c}
%\rho \uu \cdot \n\\
%\rho \uu \uu\cdot \n +  p\n\\
%\rho H \uu \cdot \n
%\end{array}
%\right],
%\text{~~}
%\F^r_c=
%%\left [ 
%%\begin{array}{c}
%%\rho \uu \cdot \n\\
%%\rho \uu \uu\cdot \n +  p\n\\
%%\rho H \uu \cdot \n
%%\end{array}
%%\right]
%\F_c
%-
%(\uu_{rot}\cdot \n) 
%\left [ 
%\begin{array}{c}
%\rho\\
%\rho \uu\\
%\rho E
%\end{array}
%\right ]
%,\text{~~}
%\F_v=
%\left [ 
%\begin{array}{c}
%0\\
%\tau \cdot \n\\
%\uu \cdot \tau \cdot \n + \kappa \n \cdot \nabla T
%\end{array}
%\right ],\text{~~}
%\F_\omega=
%\left [ 
%\begin{array}{c}
%0\\
%\rho {\boldsymbol{\omega}} \times \uu\\
%0
%\end{array}
%\right ],
%\end{equation*}
%with $\uu_{rot}={\boldsymbol \omega} \times \x$.


\subsection{Spatial discretization}
%The governing equations are discretized using the
%method of lines and thus the spatial and temporal
%discretizations can be treated separately.
%The governing equations for the
%steady state solution $\W$ is
%\begin{equation}
%\R(\W)={\bf 0},
%\label{nonlinear}
%\end{equation}
%where $\R$ is the sum of fluxes and source terms
%associated with each control volume. Suppose control
%volume %node
%$i$ has $N$ flux faces with
%area $S_{ik}$ for
%$k=1,2,...,N$.
%$R_i$ then is 
%\begin{equation*}
%R_i(\W)=\sum^{N}_{k=1} (\F^r_c-\F_v)S_{ik}+\F_{\omega} V_i,
%\end{equation*}
%where $V_i$ denotes the volume.
%
%Turbulence is modelled using the negative Spalart--Allmaras
%(SA-neg) model~\cite{allmaras2012modifications}.
%Compared to the original SA model~\cite{allmaras2012modifications},
%this avoids the clipping of the turbulent variable
%to a non-negative value which potentially
%prevents the full convergence of the nonlinear solver.
%The turbulence equation is discretized using
%the first-order accurate upwind
%scheme~\cite{langer2014agglomeration}.
%
%\subsection{Temporal discretization}
%The time-marching of the nonlinear equation is
%based on Newton's method.
%Equation~\eqref{nonlinear} is solved by iteratively updating
%the solution as
%\begin{equation*}
%  \W^{n+1}=\W^n+ \Delta \W
%  \label{newtonTransient}
%\end{equation*}
%until convergence is reached, i.e., $\|\R(\W)\|=0$,
%where $\Delta \W$ is the solution to the linear 
%system of equations
%\begin{equation*}
%  \df{\R}{\W}\Delta \W = -\R(\W^n) 
%\end{equation*}
%The exact Jacobian matrix is computed with the
%help of automatic differentiation tool
%Tapenade~\cite{Tapenade}
%and the graph coloring package
%Colpack~\cite{gebremedhin2013colpack}.
%The sparse linear system of equations arising
%from the linearization of $\R$ is solved with
%the Krylov-subspace solver, GMRES, with
%incomplete LU factorization (ILU) as the
%preconditioner. The overall algorithm
%is the Newton--Krylov (NK) method. Special care is
%taken to handle the periodic boundary condition
%when simulating flows in a single passage.

\section{Stability analysis}
\subsection{Time-domain unsteady approach}
\subsection{Eigenvalue approach}
\section{Eigenvalue analysis for large sparse matrices}

\section{Results}
\label{results}

\subsection{Laminar flow around a two-dimensional circular cylinder}

\subsubsection{Steady state calculation}
\subsubsection{Unsteady calculation}
\subsubsection{Eigenvalue analysis}

%
%We use the NutsCFD to compute the steady state laminar flow around the
%circular cylinder for Reynolds number 50, 60 and 75. Eigenvalue analysis is
%performed for each case. The three spectra is shown in Fig.~\ref{fig:cyl}.
%It can be seen that linearly unstable mode start to appear from Re=60.
%The unstable mode at Re=60 is visualized on the right in Fig.~\ref{fig:cyl}
%where our results based on the NutsCFD solver is compared to that
%computed using Nektar++~\cite{cantwell2015nektar++}.
%
%\begin{figure}[htb]
%	\centering   
%	\includegraphics[width=0.99\textwidth]{pic/cyl.png}
%	\caption{Left: spectra computed for the laminar flow around circular cylinder
%		at different Reynolds number; right: x- and y-velocity components
%		of the unstable eigenvector computed by NutsCFD compapred with
%		that with Nektar++.}
%	\label{fig:cyl}
%\end{figure}

\subsection{Transonic buffet around a two-dimensional airfoil (NACA0012)}

\subsubsection{Steady state calculation}
\subsubsection{Unsteady calculation}
\subsubsection{Eigenvalue analysis}

\subsection{Rotating stall for an annular compressor cascade (Rotor 67)}

\subsubsection{Steady state calculation}
\subsubsection{Unsteady calculation}
\subsubsection{Eigenvalue analysis}

\subsection{Rotating instability for an axial compressor rotor (Rotor 67)}

\subsubsection{Steady state calculation}
\subsubsection{Unsteady calculation}
\subsubsection{Eigenvalue analysis}

\section{Conclusion}
\label{conclusion}

\section*{Acknowledgements}

\bibliography{xu}

\end{document}
